% !Mode:: "TeX:UTF-8"

\heusetup{
  %******************************
  % 注意:
  %   1. 配置里面不要出现空行
  %   2. 不需要的配置信息可以删除
  %******************************
  %
  %=========
  % 秘级编号
  %=========
  statesecrets={公开},          %密级
  cnumber={no9527},             %编号
  natclassifiedindex={TM301.2}, %分类号
  intclassifiedindex={62-5},    %UDC编号
  %
  %=========
  % 中文信息
  %=========
  ctitlecover={基于~\LaTeX~的哈尔滨工程大学本硕博\\论文模板使用说明}, %放在封面中使用,在需要换行的地方插入\\
  ctitle={基于~\LaTeX~的哈尔滨工程大学本硕博论文模板使用说明},        %页眉使用论文标题,不换行
  cxueke={工学},                    %学位类型:工学/专业学位/工商管理/……
  csubject={动力工程及工程热物理},  %专业
  caffil={动力与能源工程学院},      %所在单位/学院
  cauthor={马冬梅},                 %论文作者
  csupervisor={孔夫子\ 教授},       %指导老师
  cassosupervisor={丹丘生\ 研究员}, % 副指导老师/专业学位企业导师
  creviewer={岑夫子\ 教授},         %论文主审人,硕士学位封面,不需要可注释掉
  % 日期自动使用当前时间,若需指定按如下方式修改:
  %cdate={超新星纪元}, %封面日期,不指定使用当前日期
  csubmitdate={20XX年XX月}, %提交日期
  coralexdate={20XX年XX月}, %答辩日期
  cstudentid={202103123},   %学号
  cstudenttype={工程硕士},  %专业学位类型
  %
  %
  %=========
  % 英文信息
  %=========
  etitle={Manual of~ \LaTeX ~Thesis Template of\\ Harbin Engineering University}, %论文题目(英文),在需要换行的地方插入\\
  exueke={Engineering}, %学科名称(英文)
  esubject={Power and Thermophysics Engineering}, %专业名称(英文)
  eaffil={School of Power and Energy Engineering}, %学院名称(英文)
  eauthor={Ma Dongmei},    %作者名称(英文)
  esupervisor={Professor KONG Fuzi}, %导师名称(英文)
  eassosupervisor={Professor DAN Qingsheng}, %企业导师名称(英文)
  % 日期自动生成,若需指定按如下方式修改:
  % edate={December, 2017},
  esubmitdate={August, 20XX},  %提交日期(英文)
  eoralexdate={December, 20XX}, %答辩日期(英文)
  estudenttype={Master of Engineering}, %专业学位类型(英文)
  %
  % 中英文关键词,用“英文逗号(,)”分割
  ckeywords={\TeX, \LaTeX, 论文, 模板},
  ekeywords={\TeX, \LaTeX, template, thesis},
}

\begin{cabstract}

  摘要的字数(以汉字计),本科、硕士学位论文一般为500 $\sim$ 1000字,博士学位论文为1000 $\sim$ 2000字,
  均以能将规定内容阐述清楚为原则,文字要精练,段落衔接要流畅。摘要页不需写出论文题目。
  英文摘要与中文摘要的内容应一致,在语法、用词上应准确无误,语言简练通顺。

  关键词是为了文献标引工作、用以表示全文主要内容信息的单词或术语。关键词不超过 5
  个,每个关键词中间用分号分隔。(关键词分隔符不用考虑,模板会自动处理。英文关键词同理。)
\end{cabstract}

\begin{eabstract}
  An abstract of a dissertation is a summary and extraction of research work
  and contributions. Included in an abstract should be description of research
  topic and research objective, brief introduction to methodology and research
  process, and summarization of conclusion and contributions of the
  research. An abstract should be characterized by independence and clarity and
  carry identical information with the dissertation. It should be such that the
  general idea and major contributions of the dissertation are conveyed without
  reading the dissertation.

  An abstract should be concise and to the point. It is a misunderstanding to
  make an abstract an outline of the dissertation and words ``the first
  chapter'', ``the second chapter'' and the like should be avoided in the
  abstract.

  Key words are terms used in a dissertation for indexing, reflecting core
  information of the dissertation. An abstract may contain a maximum of 5 key
  words, with semi-colons used in between to separate one another.
\end{eabstract}

