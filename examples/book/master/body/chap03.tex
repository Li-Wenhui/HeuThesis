%! TEX program = xelatex
%! TEX root = ../main.tex
%! TEX encoding = utf-8

%%%%%%%%%%%%%%%%%%%%%%%%%%%%%%%%%%%%%%%%%%%%%%%%%%%%%%%%%%%%%%%%%%%%%%
%
%  哈尔滨工程大学学位论文 XeLaTeX 模版 —— 正文文件 chap03.tex
%
%  版本:1.0.0
%  最后更新:
%  修改者:Leo LiWenhui lwh@hrbeu.edu.cn
%  修订者:
%  编译环境1:Ubuntu 12.04 + TeXLive 2013/2014
%  编译环境2:Windows 7/8  + TeXLive 2013/2014
%
%%%%%%%%%%%%%%%%%%%%%%%%%%%%%%%%%%%%%%%%%%%%%%%%%%%%%%%%%%%%%%%%%%%%%

\chapter{模版使用说明}[Using of the Template]
\label{chap03}

\section{个人信息}[Author Information]

使用模版的第一步是修改您的个人信息。与个人信息有关的内容位
于~{/front/cover.tex}~文件中。对照着模版内容改就好了,没有什么难度。填
写专业、姓名和导师的时候注意添加适当空格,也就是$\sim$字符,以调整对齐方式。
论文封面日期默认为最后一次编译~main.tex~的日期,论文提交日期和答辩日期需要手工设置。

\section{模版设置}[Setting of Template]

模板设置包括选择论文的学位类型、学科类型、汉字库和打印方式等,
这些内容的设置在~main.tex~文件中通过修改~\texttt{\textbackslash documentclass[]}~配置选项实现。

学位论文类型可以是:博士\texttt{doctor}、硕士\texttt{master}和学士(本科)\texttt{bachelor}几种类型。

对于专业学位硕士研究生,还需要声明专业学位选项并设置专业学位论文类型,例如:

~\texttt{\textbackslash documentclass[type = master, profdegree = true, research = applicationResearch]{heuthesisbook}}

专业学位论文类型分为四类:
\begin{itemize}
  \item \texttt{productdevelopment}:产品设计
  \item \texttt{projectdplanning}:工程规划
  \item \texttt{projectdesign}:工程设计
  \item \texttt{applicationresearch}:应用研究
\end{itemize}

详细设置方法与选项定义可以参考main.tex文件中的说明。

\section{中英文摘要、关键字}[Abstract and Key Word]

中英文摘要和关键字也位于~{/front/cover.tex}~文件中定义,将相应位置的内容替换成自己的即可。

这里附上对摘要和关键字的要求:
\begin{itemize}
  \item “摘要”是摘要部分的标题,不可省略。论文摘要是学位论文的缩影,文字
        要简练、明确。内容要包括目的、方法、结果和结论。单位制一律换算成国际标
        准计量单位制,除特殊情况外,数字一律用阿拉伯数码。文中不允许出现插图,
        重要的表格可以写入;
  \item 关键词请尽量用《汉语主题词表》等词表提供的规范词。中文关键词之间用全角
        分号间隔,末尾不加标点;
  \item 英文摘要和中文摘要对应,但不要逐字翻译。英文关键字使用半角分号间隔,
        末尾同样不加标点。
\end{itemize}

\section{正文}[Main Thesis Part]

正文部分包括了绪论(chap01.tex)、正文内容章节
(chap02.tex、chap03.tex、chap04.tex、……)等部分,均位于body文件夹中。

正文内容章节可以以chapXX.tex形式为文件名,使得文件名序号即为章
节序号以便于管理。当然使用其他符合系统文件命名要求的格式进行命名也可以,比如章节名称。
这些正文内容章节需要依次写入main.tex文件中,格式为:
\texttt{\textbackslash include{body/texfilename}}

论文中用到的所有的图片放在figure文件夹中,图片格式可以是~JPG、TIF、PDF、eps~等格式。

下面是研究生院对正文的要求:

正文是学位论文的主体,要着重反映学生自己的工作,要突出新的见解,例
如新思想、新观点、新规律、新研究方法、新结果等。正文一般可包括:理论分析;
试验装置和测试方法;对试验结果的分析讨论及理论计算结果的比较等。

正文要求论点正确,推理严谨,数据可靠,文字精练,条理分明,文字图表清晰整
齐,计算单位采用国务院颁布的《统一公制计量单位中文名称方案》中规定和名称。
各类单位、符号必须在论文中统一使用,外文字母必须注意大小写,正斜体。简化
字采用正式公布过的,不能自造和误写。利用别人研究成果必须附加说明。引用前
人材料必须引证原著文字。在论文的行文上,要注意语句通顺,达到科技论文所必
须具备的“正确、准确、明确”的要求。

\section{其他}[Others]

除正文之外,结论、致谢等其他内容位于back文件夹中,可根据需要进行修改、增删,
并根据实际情况对main.tex中的文件包含内容进行调整。

\section{格式设置}[Format]
一般来说,采用本模板后不需要另外使用字体、字号、颜色等文字格式设置操作,
模板会根据内容自动选用合适的格式。但在某些情况下,如果需要特殊设置字体、
字号与颜色,那么可以使用下面这些方法进行设置。

\subsection{字体设置}[Font Style]
本模板预定义的汉字字体包括:{\songti 宋体}、{\heiti 黑体}、{\kaishu 楷体}和{\fangsong 仿宋},
每种字体还包括正体、斜体、粗体,而且可以实现复合效果,例如:

\begin{flushleft}
  {
  {\songti 宋体 \textbf{加粗宋体}} \\
  {\heiti 黑体 \textbf{加粗黑体}} \\
  {\kaishu 楷体 \textbf{加粗楷体}} \\
  {\fangsong 仿宋 \textbf{加粗仿宋}} \\
  }
\end{flushleft}

设置字体的方法是在需要修改字体的文字前面加入字体定义指令,格式为\textbackslash~font,
其中\textbackslash~songti~表示宋体,\textbackslash~heiti~表示黑体,\textbackslash~kaishu~表示楷体,
\textbackslash~fangsong~表示仿宋。粗体的格式化指令为\textbackslash~textbf,
斜体的格式化指令为\textbackslash~textsl。
另外,可以用\{~~\}限定字体的设置范围,及将字体格式化指令和文字内容都放到\{~~\}内,
这样括号外面的内容格式自动恢复为以前的格式。

上面字体显示效果的实现代码为:

\begin{lstlisting}
{\songti 宋体 \textbf{加粗宋体}} \\
{\heiti 黑体 \textbf{加粗黑体}} \\
{\kai书 楷体 \textbf{加粗楷体}} \\
{\fangsong 仿宋 \textbf{加粗仿宋}} \\
\end{lstlisting}

\subsection{字号设置}[Font Size]

设置字号的方法为:
\begin{lstlisting}
\xiaowu   \textbackslash xiaowu~  小五,默认单倍行距 \\
\wuhao    \textbackslash wuhao~   五号,默认单倍行距 \\
\xiaosi   \textbackslash xiaosi~  小四,默认1.25倍行距 \\
\sihao    \textbackslash sihao~   四号,默认1.25倍行距 \\
\xiaosan  \textbackslash xiaosan~ 小三,默认1.25 倍行距 \\
\sanhao   \textbackslash sanhao~  三号,默认1.25 倍行距 \\
\xiaoer   \textbackslash xiaoer~  小二,默认1.25 倍行距 \\
\erhao    \textbackslash erhao~   二号,默认1.25 倍行距 \\
\xiaoyi   \textbackslash xiaoyi~  小一,默认1.25 倍行距 \\
\yihao    \textbackslash yihao~   一号,默认1.5  倍行距 \\
\end{lstlisting}

打印效果如下:

\begin{flushleft}
  {
    \xiaowu   \textbackslash xiaowu~  小五,默认单倍行距 \\
    \wuhao    \textbackslash wuhao~   五号,默认单倍行距 \\
    \xiaosi   \textbackslash xiaosi~  小四,默认1.25倍行距 \\
    \sihao    \textbackslash sihao~   四号,默认1.25倍行距 \\
    \xiaosan  \textbackslash xiaosan~ 小三,默认1.25 倍行距 \\
    \sanhao   \textbackslash sanhao~  三号,默认1.25 倍行距 \\
    \xiaoer   \textbackslash xiaoer~  小二,默认1.25 倍行距 \\
    \erhao    \textbackslash erhao~   二号,默认1.25 倍行距 \\
    \xiaoyi   \textbackslash xiaoyi~  小一,默认1.25 倍行距 \\
    \yihao    \textbackslash yihao~   一号,默认1.5  倍行距 \\
  }
\end{flushleft}

\subsection{颜色设置}[Color]

设置文字颜色的方法为:
\begin{lstlisting}
\definecolor {myrgb}{rgb}{0.25, 0.5, 0.25}
\definecolor {mycmyk}{cmyk}{1, 0.8, 0.2, 0.1}

\heiti \textcolor{black}  {这是预定义颜色-黑色 balck}  \\
\heiti \textcolor{red}    {这是预定义颜色-红色 red}  \\
\heiti \textcolor{blue}   {这是预定义颜色-蓝色 blue}  \\
\heiti \textcolor{yellow} {这是预定义颜色-黄色 yellow} \\
\heiti \textcolor{myrgb}  {这是自定义RGB颜色 myrgb}  \\
\heiti \textcolor{mycmyk} {这是自定义CMYK颜色 mycmyk} \\
\end{lstlisting}


文字颜色设置打印效果:
\begin{flushleft}
  \xiaosan
  {
    \definecolor {myrgb}{rgb}{0.25, 0.5, 0.25}
    \definecolor {mycmyk}{cmyk}{1, 0.8, 0.2, 0.1}

    \heiti \textcolor{black}  {这是预定义颜色-黑色 balck}  \\
    \heiti \textcolor{red}    {这是预定义颜色-红色 red}  \\
    \heiti \textcolor{blue}   {这是预定义颜色-蓝色 blue}  \\
    \heiti \textcolor{yellow} {这是预定义颜色-黄色 yellow} \\
    \heiti \textcolor{myrgb}  {这是自定义RGB颜色 myrgb}  \\
    \heiti \textcolor{mycmyk} {这是自定义CMYK颜色 mycmyk} \\
  }
\end{flushleft}

\section*{本章小结}[Brief summary]
简单介绍模板使用方法和文字格式化方法。
