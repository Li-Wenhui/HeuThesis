%! TEX program = xelatex
%! TEX root = ../main.tex
%! TEX encoding = utf-8

%%%%%%%%%%%%%%%%%%%%%%%%%%%%%%%%%%%%%%%%%%%%%%%%%%%%%%%%%%%%%%%%%%%%%%
%
%  哈尔滨工程大学学位论文 XeLaTeX 模版 —— 正文文件 chap01.tex
%
%  版本:1.0.0
%  最后更新:
%  修改者:Leo LiWenhui lwh@hrbeu.edu.cn
%  修订者:
%  编译环境1:Ubuntu 12.04 + TeXLive 2013/2014
%  编译环境2:Windows 7/8  + TeXLive 2013/2014
%
%%%%%%%%%%%%%%%%%%%%%%%%%%%%%%%%%%%%%%%%%%%%%%%%%%%%%%%%%%%%%%%%%%%%%

%% 如需英文目录,章节的定义需要改用\BiChapterhe和\BiSection等, 例如
%%   \BiChapter{中文章名称}{Chapter Name In English}
%%   \BiSection{中文节名称}{Section Name In English}
%%   \BiSubSection{中文小节名称}{SubSection Name In English}

\chapter{绪论}[Induction]
\label{chap01}

\section{引言}[Induction]

学位论文是典型的科技文献,其具有规范的科技文献排版要求,特别是理工类学位论文需要大量的公式和文档排版。因此研究如何提高学位论文编辑排版工作的效率有非常重要的现实意义。本文结合\XeLaTeX{}\footnote{读音:拉泰赫}文档编辑的特点,将Xe\LaTeX{}用在学位论文编辑排版,使用这种方法可以提高论文编辑的效率与排版质量,最大程度地降低论文排版的繁琐性。

\XeLaTeX{}是一种专业的科技文献排版语言,使用它写文档具有如下优势:
\begin{itemize}
  \item 将内容与格式分离,使人专注于内容书写;
  \item 编程化控制排版格式,工作灵活性和精确度高;
  \item 跨平台,兼容性和稳定性非常好。
\end{itemize}

本模板是在参考其他学校论文模板的基础上,根据哈尔滨工程的大学对本科生与研究生论文的格式要求制作,通过此模板,所有排版格式化工作由模板完成,使用户集中于论文的内容上。

\section{TeX~简介}[Brief Induction of TeX]

谈到\TeX{},人们首先会想起Donald E. Knuth \footnote{1960年凯斯工学院数学学士,1963年加州理工数学博士,同年留校任教。1968年跳槽到斯坦福,1974年获图灵奖,1992年退休,1995年获冯·诺依曼奖。}(1938--)。1962年Knuth开始写一本关于编译器设计的书,原计划是12章的单行本。不久Knuth觉得此书涉及的领域应该扩大,于是越写越多,一发不可收拾。1965年完成的初稿居然有3000页,据出版商估计,这些手稿印刷出来大概需要2000页。出书的计划只好改为七卷,每卷一或两章,这就是 \emph{The Art of Computer Programming} \footnote{已完成的前三卷是:\emph{Fundamental Algorithms}, \emph{Seminumerical Algorithms}, \emph{Sorting and Searching}。 第四卷 \emph{Combinatorial Algorithms} 的第一部分4A已出版,其余部分和第五卷 \emph{Syntactic Algorithms}正在写作中,预计2020年完成。第六卷 \emph{Theory of Context-free Languages}和第七卷 \emph{Compiler Techniques}尚未安排上工作日程。}。

1976年,当Knuth改写第二卷的第二版时,很郁闷地发现第一卷的铅版不见了;而当时数字排版刚刚兴起,质量还差强人意。于是Knuth决定自己开发一个全新的排版系统,这就是 \TeX。

1978年 \TeX 第一版发布后好评如潮,Knuth趁热打铁在1982年发布了第二版,1989年发布的 \TeX{} 3.0将7位字符改为8位。之后Knuth宣布除了修正漏洞停止 \TeX 的开发,因为它已经很稳定,而且他要集中精力完成那部巨著的后几卷。

从那时起,每发布一个修正版,版本号就增加一位小数,趋近于$\pi$;当前版本是2008年的3.1415926。他的另一个软件METAFONT的版本号趋近于$e$,目前是2.718281。Knuth希望在他离世时,\TeX 和METAFONT的版本号永远固定下来,从此人们不再改动他的代码。

\subsection{格式}[Format]

\TeX 是一种语言也是一个排版引擎 (engine) ,引擎的基本功能就是把字排成行,把行排成页,涉及到断字、断行、分页等算法。基本的 \TeX 系统只有300多个元命令 (primitive) ,十分精悍,但是很难读懂,只适于非正常人类。所以Knuth提供了一种格式 (format,宏命令的集合) 对 \TeX 进行了封装,这就是Plain \TeX ,包含600多个宏命令,然而它还是不够高级。

1980年代初期,斯坦福研究所的Leslie Lamport (1941--)\footnote{1970年加入麻省计算机同伙公司。2008年获冯·诺依曼奖。} 开发了一种新的格式,也就是 \LaTeX。1992年 \LaTeX{} 2.09发布后,Lamport退居二线,之后的开发活动由Frank Mittelbach 等人接管。他们发布的最后版本是1994年的 \LaTeXe,\LaTeX 3 的开发也在进行中,只是正式版看起来遥遥无期。

\subsection{宏包}[Package]

\LaTeX 出现之后,在它的基础上出现了很多宏包 (package) 。起初,美国数学学会 {} 看着 \TeX 是好的,就派Michael D. Spivak (1940--)\footnote{1964年普林斯顿数学博士。} 开发基于Plain \TeX 的宏包AmS\TeX{},它的开发进行了两年 (1983--1985) 。后来与时俱进的AMS又看着 \LaTeX 是好的,就想转移阵地,但是他们的字体遇到了麻烦。

恰好Mittelbach和Rainer Schöpf\footnote{\LaTeX 3的开发者之一。} 刚刚搞了个字体系统new font selection scheme for \LaTeX{} (NFSS) ,AMS看着还不错,就拜托他们把AMSFonts加入 \LaTeX,继而在1989年请他们开发\AmS\LaTeX{}。次年\AmS\LaTeX 正式发布,之后它被整合为 \AmS 宏包。

\section{优点缺点}[Advantage and Disadvantage]

通过上节内容我们已经知道,\TeX 相对于其他标记语言有较大优势,但是在桌面印刷领域还有一种不可忽视的类别,所见即所得 (WYSIWYG) 系统,比如微软的Word。其实Word也有自己的域代码 (field code) ,只是一般用户不太了解。

一般而言,\TeX 相对于所见即所得系统有如下优点:
\begin{itemize}
  \item  高质量,它制作的版面看起来更专业,数学公式尤其赏心悦目。
  \item  结构化,它的文档结构清晰。
  \item  批处理,它的源文件是文本文件,便于批处理,虽然解释 (parse) 源文件可能很费劲。
  \item  跨平台,它几乎可以运行于所有电脑硬件和操作系统平台。
  \item  免费,多数 \TeX 软件都是免费的,虽然也有一些商业软件。
\end{itemize}

相应地,\TeX 由于其工作流程,设计原则,资源的缺乏,以及历史局限性等原因也存在一些缺陷:
\begin{itemize}
  \item  语法不如HTML和XML严谨、清晰。
  \item  制作过程繁琐,有时需要反复编译,不能直接或实时看到结果。
  \item  宏包鱼龙混杂,水准参差不齐,风格不够统一。
  \item  排版样式比较统一,但因而缺乏灵活性。
  \item  相对于商业软件,用户支持不够好,文档不完善。
\end{itemize}


\section{XeTeX简介}[Inductuon of XeTeX]

\XeTeX{}\footnote{英文发音为"zee-\TeX{}"}是一种使用Unicode的\TeX{}排版引擎,并支持一些现代字体技术,例如OpenType。其作者和维护者是Jonathan Kew,并以X11自由软件许可证发布。

虽然\XeTeX{}最初只是为Mac OS X所开发,但它现在在各主要平台上都可以运作。它原生的支持Unicode,并默认其输入文件为UTF-8编码。\XeTeX{}可以在不进行额外配置的情况下直接使用操作系统中安装的字体,因此可以直接利用OpenType,Graphite中的高级特性,例如额外的字形,花体,合字,可变的文本粗细等等。\XeTeX{}提供了对OpenType中本地排版约定(locl标签)的支持,也允许向字体传递OpenType的元标签。它亦支持使用包含特殊数学字符的Unicode字体排版数学公式,例如使用Cambria Math或Asana Math字体代替传统的\TeX{}字体。

\section*{本章小结}[Brief Summary]
\LaTeX{}简介。