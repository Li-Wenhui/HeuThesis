%! TEX program = xelatex
%! TEX root = ../main.tex
%! TEX encoding = utf-8

%%%%%%%%%%%%%%%%%%%%%%%%%%%%%%%%%%%%%%%%%%%%%%%%%%%%%%%%%%%%%%%%%%%%%%
%
%  哈尔滨工程大学学位论文 XeLaTeX 模版 —— 正文文件 chap04.tex
%
%  版本:1.0.0
%  最后更新:
%  修改者:Leo LiWenhui lwh@hrbeu.edu.cn
%  修订者:
%  编译环境1:Ubuntu 12.04 + TeXLive 2013/2014
%  编译环境2:Windows 7/8  + TeXLive 2013/2014
%
%%%%%%%%%%%%%%%%%%%%%%%%%%%%%%%%%%%%%%%%%%%%%%%%%%%%%%%%%%%%%%%%%%%%%

\chapter{参考文献}[Reference]
\label{chap07}

\section{参考文献的引用}[Usage of Reference]

参考文献的引用一般有两种方式,即行间引用和上标引用。

行间引用使用~\verb|\inlinecite{引用词}|~语句实现,其显示效果是这样的:文献\inlinecite{DXM2005}论述了什么什么,而文献\inlinecite{OJP1999,kelton2002,strawderman2001,LQL1999}则对这个那个进行了研究。

其实现方式为:
\texttt{文献\cs inlinecite\{DXM2005\}论述了什么什么,而文献\cs inlinecite\{OJP1999,kelton2002,strawderman2001,LQL1999\}则对这个那个进行了研究。}

也可以使用模板预定义的简化形式~\verb|\lcite|~替代~\verb|\inlinecite|。例如:文献\lcite{DXM2005}论述了什么什么,而文献\lcite{OJP1999,kelton2002,strawderman2001,LQL1999}则对这个那个进行了研究。

其实现方式为:
\texttt{文献\cs lcite\{DXM2005\}论述了什么什么,而文献\cs lcite\{OJP1999,kelton2002,strawderman2001,LQL1999\}则对这个那个进行了研究。}

上标引用使用~\verb|\cite{引用词}|~语句实现,下面这段文字是普通的上标引用格式。

我们的一切知识都是从经验开始\cite{LQL1999},这是没有任何怀疑的\cite{DXM2005}\cite{DXM2000};
因为,如果不是对象激动我们的感官,一则由它们自己引起表象,一则使我们的知性活动运作起来,对这些表象加
以比较,把它们粘结或分开\cite{OJP1999,OJP1991},这样把感性印象的原始素材加工成称之为经验的对象
知识,那么知识能力又该由什么来唤起活动呢\cite{braun2007,kelton2002,strawderman2001,LQL1999}?所以
按照时间,我们没有任何知识是先行于经验的,一切知识都是从经验开始的。

文献标志~\verb|\inlinecite{引用词}、\cite{引用词}|~中的“引用词”也称为~\verb|citekey|。
对于参考文献\cite{OJP1999},bib文件中文献定义是这样的:
\begin{lstlisting}
  @article{ LQL1999 ,
    title={ 康德何以步安瑟尔谟的后尘? },
    author={ 李秋零 },
    journal={ 中国人民大学学报 },
    volume={2},
    year={1999}
  }
\end{lstlisting}

其中文献类型~\verb|@article{|~后的第一个数据~\verb|LQL1999|~就是“引用词”(\verb|citekey|)。
在参考文献数据库~\texttt{reference.bib}~中,每一条文献都有这样一个唯一引用词标识。

\texttt{reference.bib}~文献数据库文件是一个纯文本格式的~BibTeX~文献数据库文件,可以使用任何文本编辑器进行编辑,
也可以使用专用的软件如 ~\texttt{JabRef}~进行编辑、管理,很多商业文献管理软件如~\texttt{EndNote}~、
~\texttt{RefWorks}~、~\texttt{NoteExpress}~等也都可以导出~\texttt{BibTeX}~格式的文献数据库。

\section{~BibTeX~文献文件的写法}[BibTeX Reference]

用在~\LaTeX~中的~\textsc{Bib}\kern-.08em\TeX~文献文件的扩展名为~bib,
此模板中,该文件即为~reference.bib。bibtex.exe 命令根据~GBT7714-2015~文件定义的文献格式,
将~reference.bib 中的文献数据转换为输出文档中的文献列表。

bib~文件的编写方法可参考模板中已给出的例子,
也可参考~\href{http://bbs.ctex.org/attachment.php?aid=MTk3OTd8NjY1ODc5OGV8MTMyNTY0MTEyMnxhZGZkYWpsa0I2RGZwNDR5Z1lyeStjb1dKRS8rTnJub3lvT2FkNDNJbHl1UWVkVQ\%3D\%3D}{GBT7714-2015.bst说明文档} 中所给出的例子。


heuthesis.bst 对于国标~GB/T 7714-2015 的文献分类如表~\ref{tab:entrytypes} 所示。对于每种文献类型的缺省类型,
已经设置好相应的文献标识码,因此不需要输入相应的文献标识码。

\begin{table}[htbp]
\bicaption[tab:entrytypes]{}{GBT7714-2015.bst 的分类方式}{Table$\!$}{Classification method of GBT7714-2015.bst}
\vspace{0.5em}\centering\wuhao
\begin{tabular}{llll}
\toprule[1.5pt]
文献类型 & 缺省类型 & 扩展类型(需要手 & 主要特征\\
 &  & 工加入文献标识码) & \\
\midrule[1pt]
article & 文章[J] & 报纸中的析出文献[N] & 年,卷(期):页码\\
 &  & 在线文章[J/OL] & \\
book & 书[M] & 论文集、会议录[C] & \\
 &  & 在线书[M/OL] & \\
 &  & 汇编[G] & \\
inbook & 书的某几页[M] &  & \\
incollection & 书中析出的文章[M] & 汇编的析出文献[G] & 析出文献[文献标识码]\\
 &  & 标准的析出文献[S] & \\
proceedings &  &  & \\
inproceedings & 论文集、会议录中的 & 在线论文集、 & 析出文献[文献标识码]\\
/conference & 析出文献[C] & 会议录[C/OL] & \\
mastersthesis & 毕业论文[D] &  & 类似book类\\
phdthesis & 毕业论文[D] &  & 类似book类\\
techreport & 科技报告[R] &  & 类似book类\\
misc &  & 杂项[],例如:专利[P] & 此类一般是网上文件,\\
 &  & 网上专利[P/OL] & 按照国标规定顺序\\
 &  & 网上电子公告[EB/OL] & 编码制时不输出年份\\
 &  & 磁盘[CP/DK] & \\
\bottomrule[1.5pt]
\end{tabular}
\end{table}

\section*{本章小结}[Breif Summary]
参考文献排版方法介绍。
