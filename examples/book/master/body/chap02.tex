%! TEX program = xelatex
%! TEX root = ../main.tex
%! TEX encoding = utf-8

%%%%%%%%%%%%%%%%%%%%%%%%%%%%%%%%%%%%%%%%%%%%%%%%%%%%%%%%%%%%%%%%%%%%%%
%
%  哈尔滨工程大学学位论文 XeLaTeX 模版 —— 正文文件 chap02.tex
%
%  版本:1.0.0
%  最后更新:
%  修改者:Leo LiWenhui lwh@hrbeu.edu.cn
%  修订者:
%  编译环境1:Ubuntu 12.04 + TeXLive 2013/2014
%  编译环境2:Windows 7/8  + TeXLive 2013/2014
%
%%%%%%%%%%%%%%%%%%%%%%%%%%%%%%%%%%%%%%%%%%%%%%%%%%%%%%%%%%%%%%%%%%%%%

\chapter{XeLaTeX环境配置}
\label{chap02}

\TeX{}~可以在Windows、Linux以及MacOS等操作系统下运行,鉴于大部分人都是使用Windows或Linux类操作系统,本文主要介绍着两类操作系统下的 \TeX{} 工作环境配置。

\section{Windows~操作系统}

\subsection{安装配置}
在Windows下可以使用的 \TeX{}套件有很多种,常用的有C \TeX{}和 \TeX{}Live。建议选择这两个套件中的一个使用。其中C \TeX{}只能在Windows系统下使用,而 \TeX{}Live则可以在Windows或Linux系统下使用。这两个套件都可以在网上免费下载到,建议大家下载最新的完整版本安装,因为本论文模板使用的某些宏包比较新,不然可能会造成编译错误。

\subsection{编译运行}
如果使用C \TeX{}套件的完整版,安装程序会自动配置好必须的环境变量,安装结束就可以直接使用了。

默认的,C \TeX{}安装包中会带有WinEdt软件,这是一个非常不错的 \TeX{}编辑工具。

需要注意的是,在WinEdt中必须在每个tex文件的开始添加如下的两行:
\begin{lstlisting}
  % !TEX TS-program = XeLaTeX
  % !TEX encoding = utf-8
\end{lstlisting}
否则文件会变成乱码。

除了WinEdt之外,还有很多其他优秀的编辑器可用于tex文件的编辑,例如TeXStudio。

以本模版为例,在Windows下的编译过程是这样的:
\begin{enumerate}
\item 打开main.tex文件;
\item 先点击WinEdt工具栏上的\XeLaTeX{}按钮(可能在Acrobat Reader按钮的下拉菜单
  中);
\item 再点击WinEdt工具栏上的Bib\TeX{}按钮;
\item 再点击WinEdt工具栏上的\XeLaTeX{}按钮两到三遍;
\item 最后点击WinEdt工具栏上的Acrobat Reader按钮就可以看到输出的PDF文档了。
\end{enumerate}

\section{Linux~操作系统(以~Ubuntu~为例)}
First things first,首先的工作是安装一个合适的\XeTeX{}编译系统。这个问题
并不难解决,现在主流的\LaTeX{}编译系统均已经包含了对\XeTeX{}的支持(包
括xeCJK中文宏包),并不需要自己额外再进行安装。在Linux下推荐使
用\TeX{}Live,目前最新版本为\TeX{}Live 2021。下面以在Ubuntu下的本地安装为
例,简要的说明\TeX{}Live的安装及配置过程,高玩们请主动绕行:

\begin{enumerate}
\item 下载\TeX{}live 2021镜像,点击\href{http://ftp.ctex.org/mirrors/CTAN/systems/texlive/Images/}{这里}进
  入下载列表。如果你有检查文件完整性的习惯的话,这个列表还提供了md5和sha256校验值;
\item 安装perl-tk包,以便使用图形界面进行安装。在终端中输入命
  令\texttt{\footnotesize sudo apt-get install perl-tk};
\item 挂载下载好的iso镜像,\texttt{\footnotesize sudo mkdir
    /mnt/texlive}(在~{/mnt}~下创建texlive文件夹
  ),\texttt{\footnotesize sudo mount -o loop texlive2021.iso
    /mnt/texlive}(挂载texlive2021.iso)。进入~/mnt/texlive~目录,输入命
  令~\texttt{\footnotesize sudo ./install-tl -gui}~之后出现图形界面。之后
  的操作就比较简单了,可以去掉不用的语言包以节省磁盘空间,注意选择最后一
  项Create symlinks in system directories,让安装程序自动创建语法链接。确
  定安装,耐心等待进度条到头;
\item 配置环境变量。在终端中输入~\texttt{\footnotesize sudo gedit
    /etc/bash.bashrc},在此文件末尾添加

  \begin{lstlisting}
    PATH=/usr/local/texlive/2021/bin/i386-linux: $PATH;
    export PATH
    MANPATH=/usr/local/texlive/2021/texmf/doc/man: $MANPATH;
    export MANPATH
    INFOPATH=/usr/local/texlive/2021/texmf/doc/info: $INFOPATH;
    export INFOPATH
  \end{lstlisting}

  在~{/etc/manpath.config}~文件的~\texttt{\footnotesize\# set up PATH to
    MANPATH mapping}~这行下面的列表后增加一条:
  \begin{lstlisting}
    MANPATH_MAP /usr/local/texlive/2021/bin/i386-linux
    /usr/local/texlive/2021/texmf/doc/man
  \end{lstlisting}

  在~{/etc/manpath.config}~文件的~\texttt{\footnotesize\# set up PATH to
    MANPATH mapping}~这行下面的列表后增加一条:
  \begin{lstlisting}
    MANPATH_MAP /usr/local/texlive/2021/bin/i386-linux
    /usr/local/texlive/2021/texmf/doc/man
  \end{lstlisting}
\end{enumerate}
至此安装过程结束。

如果是在Windows系统下,可直接将Texlive可执行文件加入系统 PATH 环境变量中。

以上\TeX{}Live安装过程摘自某位筒子的博客文摘,原始链接位于wordpress空间,
访问有问题,不过
\href{http://hi.baidu.com/skubuntu/blog/item/89e8de2f73a465e08a1399a3.html}{
  百度空间}有转载,虽然百度搜不着什么玩意。

接下来我们需要安装一套中文字体,你可以使用Windows下的方正、华文或者中易字
体,但要注意选择的字体最好是包含宋体、黑体、楷体和仿宋的完整套装。如果想获得
更好的显示效果,也可以使用Adobe等其他中文字体。字体的安装及配置过程如下:

\begin{enumerate}
\item 下载Adobe中文字体,点
  击
  \href{http://forum.ubuntu.org.cn/viewtopic.php?f=35&t=180987&start=0}{
    这里}进入下载页面;
\item 将下载的字体拷至~{/usr/share/fonts/truetype/adobe}~目录,如果没有请
  以管理员身份新建;
\item 刷新字体缓存,在终端中输入~\texttt{\footnotesize sudo fc-cache -fv }。这时,你可以通过~\texttt{\footnotesize fc-list :lang=zh-cn |sort}~命令来查看字体是否安装成功,注意fc-list后有个空格;
\item 你可能还需要在终端中运行~\texttt{\footnotesize sudo apt-get
    install poppler-data cmap-adobe-cns1 cmap-adobe-gb1}命令来解决Adobe中
  文字体在PDF文件中不显示的情况。
\end{enumerate}

这样,我们就配置好了中文字体,当然这没什么特别的,网上教程一大把。

之后我们需要一个类似于WinEdt或TeXStudio的集成编译环境。在Ubuntu软件中心中,我们能很
容易的安装\TeX{}maker和\TeX{}works,两者功能差不多,\TeX{}maker更强大一些。
当然,你也可以自己配置VIM下的\LaTeX{}编译环境。在Windows环境下,可以在网上下载免费的
TeXStudio软件进行tex文件编辑。

\subsection{编译运行}

在安装并配置好编译环境之后,接下来的工作就是如何编译\XeLaTeX{}文件,生成
所需的PDF文档了。

任何文本编辑工具都可以用来编写论文,当然Linux下也有很多免费的集成编辑工具可以使用。

以本模版为例,在\TeX{}works编译过程是这样的:
\begin{enumerate}
\item 打开main.tex文件;
\item 将工具栏上的编译命令切换至\XeLaTeX{}后,点击运行;
\item 再将工具栏上的编译命令切换至Bib\TeX{}后,点击运行;
\item 再将工具栏上的编译命令切换至\XeLaTeX{}后,点击运行,这里需要运行两
  到三遍;
\item 如果编译没有错误,就可以看到输出的PDF文件了。
\end{enumerate}

对于\TeX{}maker,首先需要在【选项】【配置\TeX{}maker】【命令】中将第一行
的latex改成xelatex,之后用\LaTeX{}作为\XeLaTeX{}命令执行即可,其他的和上
面类似。

对于熟悉Linux软件开发编译流程的同学,可以为编译过程编写一个makefile后使用make工具进行处理,
对于在Windows系统下使用tex撰写论文的同学,也可以使用模板提供的build.bat批处理文件生产PDF文件。

除了正常的论文模板之外,还附带了论文书脊定义和A3封面生成工具。在生成封面前,需要编辑书脊定义spine.tex文件,将其中的论文题目和作者信息进行修改,然后执行make\_cover.bat批处理文件即可生成a3cover.pdf论文
装订用A3封皮。

模板目录中的authorization.pdf为论文原创性和授权声明,可以在生成论文前使用签字后扫描的pdf文件进行替换。

\section{字体}

可以使用Windows系统字体或者Adobe字体\footnote,但要注意选择的字体最好是包含宋体、黑体、楷体和仿宋的完整套装。

本模板默认是使用Windows库,如果使用其他字库,在使用此模板撰写论文前,应该安装相应的字库。在Windows操作系统下,只要把字库文件复制的Windows \textbackslash Fonts文件夹下即可,而对于Linux系统,可通过右键点击字库文件然后选择【安装字库】菜单选项进行安装。Linux对于系统新安装的字库,需要使用命令~sudo fc-cache -fsv:刷新缓存后才可以使用。

\section*{本章小结}
\LaTeX{}~工作环境安装与配置简介。
