%! TEX program = xelatex
%! TEX root = ../main.tex
%! TEX encoding = utf-8

%%%%%%%%%%%%%%%%%%%%%%%%%%%%%%%%%%%%%%%%%%%%%%%%%%%%%%%%%%%%%%%%%%%%%%
%
%  哈尔滨工程大学学位论文 XeLaTeX 模版 —— 正文文件 chap02.tex
%
%  版本:1.0.0
%  最后更新:
%  修改者:Leo LiWenhui lwh@hrbeu.edu.cn
%  修订者:
%  编译环境1:Ubuntu 12.04 + TeXLive 2013/2014
%  编译环境2:Windows 7/8  + TeXLive 2013/2014
%
%%%%%%%%%%%%%%%%%%%%%%%%%%%%%%%%%%%%%%%%%%%%%%%%%%%%%%%%%%%%%%%%%%%%%

\chapter{XeLaTeX环境配置}[Enverioment of XeLaTeX]
\label{chap02}

\TeX{}~可以在 Windows 、 Linux 以及 MacOS 等操作系统下运行,
鉴于大部分人都是使用 Windows 或 Linux 类操作系统,
本文主要介绍着两类操作系统下的 \TeX{} 工作环境配置。

\section{Windows~操作系统}[Windows System]

\subsection{安装配置}[Install and Config]
在 Windows 下可以使用的 \TeX{}套件有很多种,常用的有 C\TeX{}和 \TeX{}Live。
其中 C\TeX{}只能在 Windows 系统下使用,而 \TeX{}Live 则可以在 Windows 或 Linux 系统下使用。
这两个套件都可以在网上免费下载到,建议大家在 \href{https://tug.org/texlive/}{\TeX{}Live官方网站}下载最新版本的 \TeX{}Live 安装,
如果是使用光盘镜像安装,可以在安装完成后在线升级以更新宏包到最新版本。

\subsection{编译运行}[Compile]
如果使用 C\TeX{}套件的完整版,安装程序会自动配置好必须的环境变量,安装结束就可以直接使用了。

默认的,C\TeX{}安装包中会带有WinEdt软件,这是一个非常不错的 \TeX{}编辑工具。

需要注意的是,在 WinEdt 中需要在每个 tex 文件的开始添加如下的两行:
\begin{lstlisting}
  % !TEX TS-program = XeLaTeX
  % !TEX encoding = utf-8
\end{lstlisting}
否则文件可能会变成乱码。

除了 WinEdt 之外,还有很多其他优秀的编辑器可用于 tex 文件的编辑,例如 TeXStudio 。

以本模版为例,在 Windows 下的编译过程是这样的:
\begin{enumerate}
\item[(1)] 打开main.tex文件;
\item[(2)] 先点击WinEdt工具栏上的\XeLaTeX{}按钮(可能在Acrobat Reader按钮的下拉菜单中);
\item[(3)] 再点击WinEdt工具栏上的Bib\TeX{}按钮;
\item[(4)] 再点击WinEdt工具栏上的\XeLaTeX{}按钮两到三遍;
\item[(5)] 最后点击WinEdt工具栏上的Acrobat Reader按钮就可以看到输出的PDF文档了。
\end{enumerate}

\section{Linux~操作系统(以~Ubuntu~为例)}[Linux System]

\subsection{安装配置}[Install and Config]

First things first,首先的工作是安装一个合适的 \XeTeX{}编译系统。这个问题
并不难解决,现在主流的 \LaTeX{}编译系统均已经包含了对 \XeTeX{}的支持(包
括 xeCJK 中文宏包),并不需要自己额外再进行安装。在Linux下推荐使
用 \TeX{}Live ,目前最新版本为 \TeX{}Live 2021。下面以在 Ubuntu 下的本地安装为
例,简要的说明 \TeX{}Live 的安装及配置过程,高玩们请主动绕行:

\begin{enumerate}
\item[(1)] 下载 \TeX{}live 2021 镜像,点击\href{https://tug.org/texlive/}{这里}进
  入 \TeX{}live 2021 网站选择需要下载的安装文件或文档。
\item[(2)] 安装 perl-tk 包,以便使用图形界面进行安装。在终端中输入命
  令\texttt{sudo apt-get install perl-tk};
\item[(3)] 挂载下载好的iso镜像,\texttt{sudo mkdir
    /mnt/texlive}(在~{/mnt}~下创建texlive文件夹
  ),\texttt{sudo mount -o loop texlive2021.iso
    /mnt/texlive}(挂载texlive2021.iso)。进入~/mnt/texlive~目录,输入命
  令~\texttt{sudo ./install-tl -gui}~之后出现图形界面。之后
  的操作就比较简单了,可以去掉不用的语言包以节省磁盘空间,注意选择最后一
  项 Create symlinks in system directories ,让安装程序自动创建语法链接。确
  定安装,耐心等待进度条到头;
\item[(4)] 配置环境变量。在终端中输入~\texttt{sudo gedit /etc/bash.bashrc},在此文件末尾添加

  \begin{lstlisting}
    PATH=/usr/local/texlive/2021/bin/i386-linux: $PATH;
    export PATH
    MANPATH=/usr/local/texlive/2021/texmf/doc/man: $MANPATH;
    export MANPATH
    INFOPATH=/usr/local/texlive/2021/texmf/doc/info: $INFOPATH;
    export INFOPATH
  \end{lstlisting}

  在~{/etc/manpath.config}~文件的~\texttt{set up PATH to
    MANPATH mapping}~这行下面的列表后增加一条:
  \begin{lstlisting}
    MANPATH_MAP /usr/local/texlive/2021/bin/i386-linux
    /usr/local/texlive/2021/texmf/doc/man
  \end{lstlisting}

  在~{/etc/manpath.config}~文件的~\texttt{set up PATH to
    MANPATH mapping}~这行下面的列表后增加一条:
  \begin{lstlisting}
    MANPATH_MAP /usr/local/texlive/2021/bin/i386-linux
    /usr/local/texlive/2021/texmf/doc/man
  \end{lstlisting}
\end{enumerate}
至此安装过程结束。

如果是在 Windows 系统下,可直接将 Texlive 可执行文件加入系统 PATH 环境变量中。

接下来我们需要安装一套中文字体,你可以使用 Windows 系统默认字
体,但要注意选择的字体最好是包含宋体、黑体、楷体和仿宋的完整套装。如果想获得
更好的显示效果,也可以使用 Adobe 等其他中文字体。 Adobe 字体的安装及配置过程如下:

\begin{enumerate}
\item[(1)] 下载Adobe中文字体,点击\href{http://forum.ubuntu.org.cn/viewtopic.php?f=35&t=180987&start=0}{这里}进入下载页面;
\item[(2)] 将下载的字体拷至~{/usr/share/fonts/truetype/adobe}~目录,如果没有请以管理员身份新建;
\item[(3)] 刷新字体缓存,在终端中输入~\texttt{sudo fc-cache -fv }。这时,你可以通过~\texttt{fc-list:lang=zh-cn |sort}~命令来查看字体是否安装成功,注意fc-list后有个空格;
\item[(4)] 你可能还需要在终端中运行~\texttt{sudo apt-get install poppler-data cmap-adobe-cns1 cmap-adobe-gb1}命令来解决Adobe中文字体在PDF文件中不显示的情况。
\end{enumerate}

这样,我们就配置好了中文字体,当然这没什么特别的,网上教程一大把。

之后我们需要一个类似于 WinEdt 或 TeXStudio 的集成编译环境。在 Ubuntu 软件中心中,我们能很
容易的安装 \TeX{}maker 和 \TeX{}works ,两者功能差不多, \TeX{}maker 更强大一些。
当然,你也可以自己配置 VIM 下的 \LaTeX{}编译环境。在 Windows 环境下,可以在网上下载免费的
TeXStudio 软件进行 tex 文件编辑。

\subsection{生成论文}[Compile]
\label{sec:generate-thesis}

在安装并配置好编译环境之后,接下来的工作就是如何编译 \XeLaTeX{} 文件,生成
所需的 PDF 文档了。

任何文本编辑工具都可以用来编写论文,当然 Linux 下也有很多免费的集成编辑工具可以使用。

本节介绍几种常见的生成论文的方法。用户可根据自己的情况选择。

\subsubsection{\XeLaTeX}
\label{sec:xelatex}
很多用户对 \LaTeX\ 命令执行的次数不太清楚。一个基本的原则是多次运行 \LaTeX\ 命
令直至不再出现警告。下面给出生成示例文档的详细过程(\texttt{\#} 开头的行为注
释),首先来看推荐的 \texttt{xelatex} 方式:
\begin{lstlisting}
# 1. 发现里面的引用关系,文件后缀 .tex 可以省略
$ xelatex main

# 2. 编译参考文件源文件,生成 bbl 文件
$ bibtex main

# 3. 解决引用
$ xelatex main
$ xelatex main   # 如果不需要生成索引此时生成完整的 pdf 文件
$ splitindex main -- -s heuthesis.ist  # 自动生成索引
$ xelatex main.tex
\end{lstlisting}

\subsubsection{latexmk}
\label{sec:latexmk}
\texttt{latexmk} 命令支持全自动生成 \LaTeX\ 编写的文档,并且支持使用不同的工具
链来进行生成,它会自动运行多次工具直到交叉引用都被解决。下面给出了一个用
\texttt{latexmk} 调用 \texttt{xelatex} 生成最终文档的示例:
\begin{lstlisting}
$ latexmk -xelatex main
\end{lstlisting}

\subsubsection{make}
\label{sec:make}

上面使用 \texttt{xelatex} 生成论文的方法虽然不复杂,但是每次都输入还是非常罗嗦,所以 \heuthesis\ 
提供了一个 \texttt{Makefile},可以通过在命令行环境下执行一次 make 完成论文生成这些工作。

\subsubsection{buidl.bat}
\label{sec:build}

示例文件夹中也包含了 \texttt{build.bat} 批处理文件,在 \texttt{Windows} 环境下可以
直接使用改批处理文件生成最终 \texttt{PDF} 格式的论文 \texttt{main.pdf}。

\subsection{原创性和授权声明}
\label{sec:generate-auth}

按学校研究生论文规范要求,硕士和博士论文扉页后应有论文原创性和授权声明签字页。
该页可以使用两种方式实现:

\begin{enumerate}
  \item[(1)] \texttt{\cs authorization} 
  \item[(2)] \texttt{\cs authorization[auscan.pdf]} 
\end{enumerate}

格式(1)不带参数,使用模板生成的空白原创性和授权声明签字页,可先使用这种方式生成签字页后进行打印签字,然后扫描成 PDF 格式文件。

格式(2)使用带参数的方式,~\texttt{[auscan.pdf]}~指定使用~\texttt{auscan.pdf}~文件作为嵌入到论文中的原创性和授权声明签字页。

模板目录中已包含空白论文原创性和授权声明文件 auscan.pdf,可以在生成论文前使用签字后扫描的pdf文件进行替换,也可以指定其他符合系统要求的文件名称。

\subsection{生成封面}
\label{sec:generate-cover}
模板还提供了用于生成论文 \texttt{A3} 页面的封面定义文档。用户可根据自己的情况选择。

A3 封面的生成是利用上述论文的首页封面和书脊拼合而成,因此在生成 A3 封面前
需要先生成论文 \texttt{main.pdf} 并在 \texttt{spine.tex} 定义论文标题和作者姓名信息,
然后使用 \texttt{xelatex} 依次编译 \texttt{spine.tex} 和 \texttt{a3cover.tex}:

\begin{lstlisting}
# 1. 使用 xelatex 
$ xelatex spine
$ xelatex a3cover

# 2. 或者使用 latexmk
$ latexmk spine
$ latexmk a3cover
\end{lstlisting}

在 \texttt{Windows} 系统下也可以直接使用 \texttt{make\_cover.bat} 批处理生成 \texttt{A3} 封面。

\section{字库安装}[Font Install]

可以通过模板选项选择~\texttt{fontset}~使用Windows系统字体或者Adobe字体。

本模板默认是使用Windows库,可以不用设置字体选项,如果使用Adobe字体字库,需要设置~\texttt{fontset=adobe}~。
在使用此模板撰写论文前,应该确保相应的字库已经安装,并且最好是包含宋体、黑体、楷体和仿宋的完整套装。
在 Windows 操作系统下,只要把字库文件复制到 Windows 的 Fonts 文件夹下即可,
而对于 Linux 系统,可通过右键点击字库文件然后选择【安装字库】菜单选项进行安装。
Linux 对于系统新安装的字库,需要使用命令~\texttt{sudo fc-cache -fsv}~刷新缓存后才可以使用。

\section*{本章小结}[Brief Summary]
\LaTeX{}~工作环境安装与配置简介。
