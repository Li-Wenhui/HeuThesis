% !Mode:: "TeX:UTF-8"

\chapter[哈尔滨工程大学研究生学位论文撰写规范]{哈尔滨工程大学研究生学
  位论文\protect\\撰写规范}[Harbin Engineering University Postgraduate Dissertation Writing Specifications]

研究生学位论文是研究生科学研究工作的全面总结,是描述其研究成果、代表其研究水平的
重要学术文献资料,是申请和授予相应学位的基本依据。学位论文撰写是研究生培养过程的
基本训练之一,必须按照确定的规范认真执行。研究生应严肃认真地撰写学位论文,指导教
师应加强指导,严格把关。

学位论文撰写应实事求是,杜绝造假和抄袭等行为;应符合国家及各专业部门制定的有关标
准,符合汉语语法规范。

硕士和博士学位论文,除在字数、理论研究的深度及创造性成果等
方面的要求不同外,撰写规范要求基本一致。人文与社会科学、管理学科可在本撰写规范的
基础上补充制定专业的学术规范。

\section{内容要求}[Content specification]

\subsection{题目}[Title]

题目应以简明的词语,恰当、准确、科学地反映论文最重要的特定内容(一般不超过25字),
应中英文对照。题目通常由名词性短语构成,不能含有标点符号;应尽量避免使用不常用的
缩略词、首字母缩写字、字符、代号和公式等。

如题目内容层次很多,难以简化时,可采用题目和副题目相结合的方法。
题目与副题目字数之和不应超过35字,中文的题目与副题目之间用破折号相连,英文则用冒
号相连,\emph{除此之外不能含有标点符号}。
副题目起补充、阐明题目的作用。题目和副题目在整篇学位论文中的不同地方出现时,应保持一致。

\subsection{摘要与关键词}[Abstraction and key words]
\subsubsection{摘要}[Abstraction]

摘要是论文内容的高度概括,应具有独立性和自含性,即不阅读论文的全文,就能通过摘要
了解整个论文的必要信息。摘要应包括本论文研究的目的、理论与实际意义、主要研究内容、
研究方法等,重点突出研究成果和结论。

摘要的内容要完整、客观、准确,应做到不遗漏、不拔高、不添加。摘要应按层次逐段简要
写出,避免将摘要写成目录式的内容介绍。摘要在叙述研究内容、研究方法和主要结论时,
除作者的价值和经验判断可以使用第一人称外,一般使用第三人称,采用“分析了……原因”、
“认为……”、“对……进行了探讨”等记述方法进行描述。避免主观性的评价意见,避免
对背景、目的、意义、概念和一般性(常识性)理论叙述过多。

摘要需采用规范的名词术语(包括地名、机构名和人名)。对个别新术语或无中文译文的术
语,可用外文或在中文译文后加括号注明外文。摘要中不宜使用公式、化学结构式、图表、
非常用的缩写词和非公知公用的符号与术语,不标注引用文献编号。

博士学位论文摘要应包括以下几个方面的内容:

(1)论文的研究背景及目的。简洁准确地交代论文的研究背景与意义、相关领域的研究现
状、论文所针对的关键科学问题,使读者把握论文选题的必要性和重要性。此部分介绍不宜
写得过多,一般不多于400字。

(2)论文的主要研究内容。介绍论文所要解决核心问题开展的主要研究工作以及研究方法
或研究手段,使读者可以了解论文的研究思路、研究方案、研究方法或手段的合理性与先进
性。

(3)论文的主要创新成果。简要阐述论文的新思想、新观点、新技术、新方法、新结论等
主要信息,使读者可以了解论文的创新性。

(4)论文成果的理论和实际意义。客观、简要地介绍论文成果的理论和实际意义,使读者
可以快速获得论文的学术价值。

\subsubsection{关键词}[Keywords]
关键词是供检索用的主题词条。关键词应集中体现论文特色,反映研究成果的内涵,具有语
义性,在论文中有明确的出处,并应尽量采用《汉语主题词表》或各专业主题词表提供的规
范词,应列取3$\sim$6个关键词,按词条的外延层次从大到小排列。

\subsection{目录}[Content]

论文中各章节的顺序排列表,包括论文中全部章、节、条三级标题及其页码。

\subsection{论文正文}[Main body]

论文正文包括绪论、论文主体及结论等部分。

\subsubsection{绪论}[Introduction]
绪论一般作为第1章。绪论应包括:本研究课题的来源、背景及其理论意义与实际意义;国
内外与课题相关研究领域的研究进展及成果、存在的不足或有待深入研究的问题,归纳出将
要开展研究的理论分析框架、研究内容、研究程序和方法。

绪论部分要注意对论文所引用国内外文献的准确标注。绪论的主要研究内容的撰写宜使用将
来时态,切忌将论文目录直接作为研究内容。

\subsubsection{论文主体}[Main body]
论文主体是学位论文的主要部分,应该结构严谨,层次清晰,重点突出,文字简练、通顺。
论文各章之间应该前后关联,构成一个有机的整体。论文给出的数据必须真实可靠,推理正
确,结论明确,无概念性和科学性错误。对于科学实验、计算机仿真的条件、实验过程、仿
真过程等需加以叙述,避免直接给出结果、曲线和结论。引用他人研究成果或采用他人成说
时,应注明出处,不得将其与本人提出的理论分析混淆在一起。

论文主体各章后应有一节“本章小结”,实验方法或材料等章节可不写“本章小结”。各章
小结是对各章研究内容、方法与成果的简洁准确的总结与概括,也是论文最后结论的依据。

\subsubsection{结论}[Conclusion]
结论作为学位论文正文的组成部分,单独排写,不加章标题序号,不标注引用文献。
结论内容一般在\num{2000}字以内。
结论应是作者在学位论文研究过程中所取得的创新性成果的概要总结,不能与摘要混为一谈。
博士学位论文结论应包括论文的主要结果、创新点、展望三部分,在结论中应概括论文的核
心观点,明确、客观地指出本研究内容的创新性成果(含新见解、新观点、方法创新、技术
创新、理论创新),并指出今后进一步在本研究方向进行研究工作的展望与设想。
对所取得的创新性成果应注意从定性和定量两方面给出科学、准确的评价,分(1)、(2)、
(3)…条列出,宜用“提出了”“建立了”等词叙述。
此外,结论的撰写还应符合以下基本要求:
(1)结论具有相对的独立性,不应是对论文中各章小结的简单重复。
结论要与引言相呼应,以自身的条理性、明确性、客观性反映论文价值。对论文创新内容的概括、评价要适当。
(2)结论措辞要准确、严谨,不能模棱两可,避免使用“大概”“或许”“可能是”等词
语。
结论中不应有解释性词语,而应直接给出结果。结论中一般不使用量的符号,而宜用量的名称。
(3)结论应指出论文研究工作的局限性或遗留问题,如条件所限,或存在例外情况,或本论文尚难以解释或解决的问题。
(4)常识性的结果或重复他人的结果不应作为结论。

\subsection{参考文献}[Reference]
所有被引用文献均要列入参考文献中,必须按顺序标注,但同一篇文章只用一个序号。
尽量引用原始文献。当不能引用原始文献时,要将二次引用文献、原始文献同时标注。
博士学位论文的参考文献一般不少于100篇,硕士学位论文的参考文献一般不少于40篇,其
中外文文献一般不少于总数的1/2。参考文献中近五年的文献数一般应不少于总数的1/3,并
应有近两年的参考文献。
教材、产品说明书、国家标准、未公开发表的研究报告(著名的内部报告如PB、AD报告及著
名大公司的企业技术报告等除外)等通常不宜作为参考文献引用。

引用网上参考文献时,应注明该文献的准确网页地址,网上参考文献和各类标准不包含在上
述规定的文献数量之内。
本人在攻读学位期间发表的学术论文不应列入参考文献中。

\subsection{攻读学位期间取得创新性成果}[Publications]
学位论文后应列出研究生在攻读学位期间发表的与学位论文内容相关的学术论文(含已录用
的论文)。
攻读学位期间所获得的科研成果、专利可单做一项分别列出。
与学位论文无关的学术论文、署名为第二作者(不含第一作者为导师和副导师)以后的学术
论文,不宜在此列出。

\subsection{原创性声明及使用权限}[Authorization]
作者可直接下载本部分内容电子版。作者和导师本人签署姓名。

\subsection{致谢}[Acknowledgments]
对导师和给予指导或协助完成学位论文工作的组织和个人,对课题给予资助者表示感谢。内容应简朴、语言应含蓄。

\subsection{个人简历}[Resume]
包括学习经历和工作经历。

\section{书写规定}[Regulation]
\subsection{论文正文字数}[Word number]
博士学位论文正文一般为6万$\sim$10万字(含图表)。
硕士学位论文正文一般为3万$\sim$5万字(含图表)。
\subsection{论文书写}[Writing]
研究生学位论文一律要求在计算机上输入、编排与打印。
页码在版心下边线之下居中排放;摘要、目录、物理量名称及符号表等文前部分的页码用罗马数字单独编排,正文以后的页码用阿拉伯数字编排。
硕士学位论文的扉页、摘要,博士学位论文的扉页、摘要、目录、图题及表题等,都要求用中、英文两种文字给出,具体编排上中文在前。
留学生和外语专业的学位论文的扉页、摘要及目录,要求用中、英文两种文字给出,其他用英文或所学专业相应的语言撰写。扉页、摘要及目录的英文部分另起一页。

\subsection{摘要与关键词}[Abstract]
摘要的字数(以汉字计),硕士学位论文一般为500$\sim$\num{1000}字,博士学位论文为
\num{1000}$\sim$\num{2000}字,均以能将规定内容阐述清楚为原则,文字要精练,段落衔
接要流畅。
摘要页不需写出论文题目。
英文摘要与中文摘要的内容应完全一致,在语法、用词上应准确无误,语言简练通顺。
留学生的英文版学位论文中应有不少于\num{3000}字的“详细中文摘要”。
关键词在摘要后列出,中英文关键词应一一对应,分别排在中英文摘要下方,中文关键词之间用“;”隔开,英文关键词之间用“,”隔开。

\subsection{目录}[Contents]
目录应包括论文中全部章、节、条三级标题及其页码,含:
摘要
Abstract
物理量名称及符号表(参照附录1,采用国家标准规定符号者可略去此表)
正文章节题目(要求编到第3级标题,即×.×.×)
结论
参考文献
附录
攻读□士学位期间发表的学术论文(□为“博”或“硕”)
原创性声明
使用授权说明
索引(可选择或不选择)
致谢
个人简历

\subsection{论文正文}[Main text]
\subsubsection{章节及各章标题}[Titles]
论文正文分章节撰写,每章应另起一页。
各章节标题要突出重点、简明扼要。
字数一般应在15字以内,标题中不加标点符号。
标题中尽量不采用英文缩写词,必须采用时应使用本行业的通用缩写词。
\subsubsection{层次}[Hierarchy]
层次以少为宜,应根据实际需要选择。
层次代号建议采用本文3.7中表1的格式。
层次要求统一,若节下内容无须列条,可直接列项。具体用到哪一层次,视需要而定。
\subsection{引用文献标注}[Reference]
引用文献标注遵照《信息与文献参考文献著录规则》(GB/T7714—2015),采用顺序编码制。
正文中引用文献的标示应置于所引内容最后一个字的右上角,所引文献编号用阿拉伯数字置
于方括号“[ ]”中,用小4号字体的上角标。
要求:
(1)引用单篇文献时,如“二次铣削\cite{cnproceed}”。

(2)同一处引用多篇文献时,各篇文献的序号在方括号内全部列出,各序号间用“,”,如
遇连续序号,可标注讫序号。如,…形成了多种数学模型\cite{cnarticle,cnproceed}…
注意此处添加\cs{inlinecite}中文空格\inlinecite{cnarticle,cnproceed},可以在cfg文件中修改空格类型。

(3)多次引用同一文献时,在文献序号的“[ ]”后标注引文页码。如,…间质细胞CAMP含量
测定\cite[100-197]{cnarticle}…。…含量测定方法规定
\cite[92]{cnarticle}…。

(4)当提及的参考文献为文中直接说明时,则用小4号字与正文排齐,如“由文献\inlinecite{heuthesis2017}可知”

\subsection{名词术语}[Glossary]
科技名词术语及设备、元件的名称,应采用国家标准或部颁标准中规定的术语或名称。
标准中未规定的术语要采用行业通用术语或名称。
全文名词术语必须统一,一些特殊名词或新名词应在适当位置加以说明或注解。
采用英语缩写词时,除本行业广泛应用的通用缩写词外,文中第一次出现的缩写词应该用括号注明英文原词。

\subsection{物理量标注}[Symbols]
\subsubsection{物理量的名称和符号}[Name and symbols]
物理量的名称和符号应符合《国际单位制及其应用》(GB 3100—93)、《量和单位》(GB
3102.1$\sim$13—93)的规定。
论文中某一物理量的名称和符号应统一。
物理量的符号必须采用斜体。

\subsubsection{物理量计量单位}[Units]
物理量计量单位及符号应按国务院1984年发布的《中华人民共和国法定计量单位》(见附录
1)及《国际单位制及其应用》(GB 3100—93)、《量和单位》(GB 3102.1$\sim$13—93)
执行,不得使用非法定计量单位及符号。
计量单位可采用汉字或符号,但应前后统一。计量单位符号,除用人名命名的单位第一个字母用大写之外,一律用小写字母。
非物理量单位(如件、台、人、元、次等)可以采用汉字与单位符号混写的方式,如“万t·km”、“t/(人·a)”等。
不定数字之后可用中文计量单位符号,如“几千克”。
表达时刻时应采用中文计量单位,如“上午8点3刻”,不能写成“8h45min”。
计量单位符号一律用正体。

\subsection{外文字母的正体与斜体用法}[English]
按照《国际单位制及其应用》(GB 3100—93)、《量和单位》(GB 3102.1$\sim$13—93)的
规定,物理量符号、物理常量、变量符号用斜体,计量单位等符号用正体。外文字母采用
Times New Roman(新罗马)字体。

\subsection{数字}[Number]
按《出版物上数字用法》(GB/T 15835—2011),除习惯用中文数字表示的以外,一般均采
用阿拉伯数字(参照附录2),Times New Roman字体。

\subsection{公式}[Equation]
论文中的公式应另起行,并居中书写,与周围文字留有足够的位置区分开。公式应标注序号,
并将序号置于括号内。公式序号按章编排,如第1章第1个公式的序号为“(1-1)”。公式
的序号右端对齐。
文中引用公式时,一般用“见式(1-1)”或“由公式(1-1)”。
若公式前有文字(如“解”“假定”等),文字前空4个半角字符,公式仍居中排,公式末不加标点。
公式中用斜线表示“除”的关系时应采用括号,以免含糊不清,如。通常“乘”的关系在前,如,而不写成。
公式较长时最好在“=”(等号)处转行,如难实现,则可在“+、-、×、÷”运算符号
处转行,转行时运算符号仅书写于转行式前,不重复书写。
公式中第一次出现的物理量代号应给予注释,注释的转行应与破折号“——”后第一个字对齐。
破折号占4个半角字符,注释物理量需用公式表示时,公式后不应出现公式序号,如(3-1)。
公式中应注意分数线的长短(主、副分数线严格区分),长分数线与等号对齐,不能用文字
形式表示等式。
公式中变量下标按《量和单位》中规定,建议用正体形式。

\begin{equation}\label{eq:1}
  \ddot{\boldsymbol{\rho}}-\frac{\mu}{R_{t}^{3}}\left(3\mathbf{R_{t}}\frac{\mathbf{R_{t}\rho}}{R_{t}^{2}}-\boldsymbol{\rho}\right)=\mathbf{a}
\end{equation}
\begin{tabularx}{\textwidth}{@{}l@{\quad}r@{———}X@{}}
  式中 & $\boldsymbol{\rho}$        & 追踪飞行器与目标飞行器之间的相对位置矢量;          \\
       & $\boldsymbol{\ddot{\rho}}$ & 追踪飞行器与目标飞行器之间的相对加速度;            \\
       & $\mathbf{a}$               & 推力所产生的加速度;                                \\
       & $\mathbf{R_t}$             & 目标飞行器在惯性坐标系中的位置矢量;                \\
       & $\omega_{t}$               & 目标飞行器的轨道角速度;                            \\
       & $\mathbf{g}$               & 重力加速度,$\mathbf{g}=\frac{\mu}{R_{t}^{3}}\left(
    3\mathbf{R_{t}}\frac{\mathbf{R_{t}\rho}}{R_{t}^{2}}-\boldsymbol{\rho}\right)=\omega_{t}^{2}\frac{R_{t}}{p}\left(
    3\mathbf{R_{t}}\frac{\mathbf{R_{t}\rho}}{R_{t}^{2}}-\boldsymbol{\rho}\right)$,这里~$p$~是目标飞行器的轨道半通径。
\end{tabularx}\vspace{\ccwd}

\subsection{插表}[Table]
表应有自明性。表格不加左、右边线。表的编排建议采用国际通行的三线表,如果三线表不足以清晰表达表中内容,应加大栏与栏间距,以清晰明了为主,例如附录1中的表2。表中文字用宋体、Times New Roman字体,字号尽量采用5号字(当字数较多时可用小5号字,但在一个插表内字号要统一)。
每个表格均应有表题(由表序和表名组成)。表序一般按章编排,如第1章第一个插表的序号为“表1-1”。表序与表名之间空2个半角字符,表名中不允许使用标点符号,表名后不加标点。表题置于表上,用中、英两种文字居中排写,中文在上,用宋体5号字,英文用Times New Roman字体5号字。硕士学位论文只用中文表题。
表头设计应简单明了,尽量不用斜线。表头中可采用化学符号或物理量符号。
全表如用同一单位,则将单位符号移至表头右上角,加圆括号。
表中数据应准确无误,书写清楚。数字空缺的格内加横线“—”(占2个半角字符)。表内文字或数字上、下或左、右相同时,采用通栏处理方式,不允许用“〃”“同上”之类的写法。
表内文字说明,起行空2个半角字符,转行顶格,句末不加标点。
插表之前文中必须有相关文字提示,如“见表1-1”“如表1-1所示”。一般情况下插表不能拆开两页编排,如某表在一页内安排不下时,才可转页,以续表形式接排。表右上角注明编号,编号后加“(续表)”,并重复表头。插表的上下与文中文字间需空一行编排。
引用文献中的表格时,除在正文文字中标注参考文献序号以外,还必须在表题的右上角标注参考文献序号。
2.13  插图
图应有自明性。插图应与文字紧密配合,文图相符,内容正确。选图要力求精练,插图、照片应完整清晰。

机械工程图:采用第一角投影法,严格按照《技术制图图样画法指引线和基准线的基本规定》
(GB/T 4457.2—2003)、《机械制图机构运动简图用图形符号》(GB/T 4460—2013)、《技
术制图简化表示法》(GB/T 16675.1$\sim$2—2012)、《产品几何技术规范(GPS)技术产
品文件中表面结构的表示法》(GB/T 131—2006)及《机械工程CAD制图规则》(GB/T
14665—2012)有关规定。

数据流程图、程序流程图、系统流程图等按《信息处理 数据流程图、程序流程图、系统流
程图、程序网络图和系统资源图的文件编制符号及约定》(GB/T 1526—1989)规定。

电气图:图形符号、文字符号等应符合附录3所列有关标准的规定。

流程图:必须采用结构化程序并正确运用流程框图。

对无规定符号的图形应采用该行业的常用画法。
坐标图的坐标线均用细实线,粗细不得超过图中曲线;有数字标注的坐标图,必须注明坐标单位。
照片图要求主题和主要显示部分的轮廓鲜明,便于制版。如用放大或缩小的复制品,必须清
晰,反差适中。照片上应有表示目的物尺寸的标度。
引用文献中的图时,除在正文文字中标注参考文献序号以外,还必须在图题的右上角标注参考文献序号。

\subsubsection{图题及图中说明}[Legend]
每个图均应有图题(由图序和图名组成),图题不宜有标点符号,图名在图序之后空2个半角字符排写。图序按章编排,如第1章第一个插图的图号为“图1-1”。图题置于图下,用中、英两种文字,居中书写,中文在上,要求中文用宋体5号字,英文用Times New Roman字体5号字。有图注或其他说明时应置于图题之上。引用图应注明出处,在图题右上角加引用文献号。图中若有分图时,分图题置于分图之下或图题之下,可以只用中文书写,分图号用(a)、(b)等表示,在图题之下连续排列,用“;”间隔。
图中各部分说明应采用中文(引用的外文图除外)或数字符号,各项文字说明置于图题之上(有分图时,置于分图题之上)。
图中文字用宋体、Times New Roman字体,字号尽量采用5号字(当字数较多时可用小5号字,以清晰表达为原则,但在一个插图内字号要统一)。同一图内使用文字应统一。图表中物理量、符号用斜体。

\subsubsection{插图编排}[Figures]
插图之前,文中必须有关于本插图的提示,如“见图1-1”“如图1-1所示”等。插图与其图题为一个整体,不得拆开排写于两页。插图处的该页空白不够排写该图整体时,则可将其后文字部分提前排写,将图移到次页。有分图时,分图过多在一页内安排不下时,可转到下页,总图题只出现在该页,下页标“图序(续图)”字样。
插图的上下与文中文字间需留一定位置编排。

\subsection{参考文献}[Reference]
参考文献标注采用顺序编码制,著录格式应遵照《信息与文献  参考文献著录规则》(GB/T 7714—2015)的要求。参考文献及电子文献载体标志代码、著录细则、参考文献著录格式见附录4。以下是论文中常用的六种参考文献类型标注形式。

(1)图书文献。

[1]唐绪军. 报业经济与报业经营[M]. 北京:新华出版社,1999:117-121.

[2]霍斯尼 R K. 谷物科学与工艺学原理[M]. 李庆龙,译. 北京:中国仪器出版社,1989:32-35.

(2)期刊论文。

[1]覃睿,田先钰. 从创新潜力到创新成果:一个创新潜力形成与释放模型[J]. 科技进步与对策,2007(2):148-152.

(3)学术会议。

[1]张佐光,张晓宏,仲伟虹,等. 多相混杂纤维复合材料拉伸行为分析[C]//第九届全国复合材料学术会议论文集(下册). 北京:世界图书出版公司,1996:410-416.

(4)学位论文。

[1]金宏. 导航系统的精度及容错性能的研究[D]. 北京:北京航空航天大学,1998:60-63.

(5)电子文献。

[1] 数字化转型 2.0 时代,未来的人才与组织要如何定义?[EB/OL]( 2019-10-24) [2020-01-02]. http: // www.chinatradenews. com.cn / content /201910 /24 / c87965.html.

(6)报告。

[1]  中国互联网信息中心.第45次中国互联网络发展情况统计报告[R]. 中华人民 共和国国家互联网信息办公室,2020:1.

\subsection{附录}[Appendix]
附录作为主体部分的补充,并不是必需的。
下列内容可以作为附录置于论文后:
(1)为了整篇论文材料的完整,但编入正文又有损于编排的条理性和逻辑性,这一材料包括比正文更为详尽的信息、研究方法和技术更深入的叙述,对了解正文内容有用的补充信息等。
(2)由于篇幅过大或取材于复制品而不便于编入正文的材料。
(3)不便于编入正文的罕见珍贵资料。
(4)对一般读者并非必要阅读,但对本专业同行有参考价值的资料。
(5)某些重要的原始数据、数学推导、结构图、统计表、自编的计算机程序、计算机打印输出件等。

\subsection{攻读学位期间取得创新性成果}[Publications]
书写格式与参考文献相同,页码后需注明该文章对应学位论文的章节序号。如已发表的学术论文被EI或SCI收录,应标明收录号;SCI论文一般应标注发表当年的影响因子;对已录用但尚未发表的学术论文,请注明是否EI或SCI刊源。

\subsection{索引}[Index]
为便于检索文中内容,可编制索引置于论文之后(根据需要决定是否设置)。索引以论文中的专业词语为检索线索,指出其相关内容的所在页码。索引用中、英两种文字书写,中文在前。中文按各词汉语拼音第一个字母排序,英文按该词第一个英文字母排序。索引示例见附录5。
\subsection{个人简历}[Resume]
除全日制硕士生外,其余学生均增列此项。个人简历一般应包含大学起的学习经历和工作经历。

\subsection{书脊}[Ridge]
为了便于学位论文的管理,建议参照《图书和其它出版物的书脊规则》(GB/T 11668—1989)规定,在学位论文书脊中标注学位论文题目及学位授予单位名称,用小4号黑体。

\subsection{其他}[Else]
年代前必须注明世纪,如20世纪70年代。


% Local Variables:
% TeX-master: "../thesis"
% TeX-engine: xetex
% End: