
\appendix{B}{中华人民共和国法定计量单位}%[Units]

\section{法定计量单位}
我国的法定计量单位(简称法定单位)包括:

(1)国际单位制的基本单位(见表1);

(2)国际单位制的辅助单位(见表2);

(3)国际单位制中具有专门名称的导出单位(见表3);

(4)国家选定的非国际单位制单位(见表4);

(5)由以上单位构成的组合形式的单位;

(6)由词头和以上单位所构成的十进倍数和分数单位(词头见表5)。法定单位的定义、使用方法等,由国家计量局另行规定。 

\hspace*{\fill} \\

\section{基本单位}

\begin{table}[htbp]
  \centering
  \caption{国际单位制的基本单位}
  \begin{tabular}{ccc}
    \toprule
    量的名称  & 单位名称   & 单位称号 \\
    \midrule
    长度    & 米      & m    \\
    质量    & 千克(公斤) & kg   \\
    时间    & 秒      & s    \\
    电流    & 安[培]   & A    \\
    热力学温度 & 开[尔文]  & K    \\
    物质的量  & 摩[尔]   & mol  \\
    发光强度  & 坎[德拉]  & cd   \\
    \bottomrule
  \end{tabular}%
  \label{tab:tbl-b1}%
\end{table}%

\hspace*{\fill} \\

\section{辅助单位}

\begin{table}[htbp]
  \centering
  \caption{国际单位制的辅助单位}
  \begin{tabular}{ccc}
    \toprule
    量的名称 & 单位名称 & 单位符号 \\
    \midrule
    平面角  & 弧 度  & rad  \\
    立体角  & 球面度  & sr   \\
    \bottomrule
  \end{tabular}%
  \label{tab:tbl-b2}%
\end{table}%

\hspace*{\fill} \\

\section{导出单位}

\begin{table}[htbp]
  \centering
  \caption{国际单位制中具有专门名称的导出单位}
  \begin{tabular}{cccc}
    \toprule
    量的名称        & 单位名称   & 单位符号 & 其他表示式例                    \\
    \midrule
    频率          & 赫[兹]   & Hz   & S\textsuperscript{-1}     \\
    力;重力        & 牛[顿]   & N    & kg·m/s\textsuperscript{2} \\
    压力;压强;应力    & 帕[斯卡]  & Pa   & N/m\textsuperscript{2}    \\
    能量;功;热      & 焦[耳]   & J    & N·m                       \\
    功率;辐射通量     & 瓦[特]   & W    & J/s                       \\
    电荷量         & 库[仑]   & C    & A·s                       \\
    电位;电压;电动势   & 伏[特]   & V    & W/A                       \\
    电容          & 法[拉]   & F    & C/V                       \\
    电阻          & 欧[姆]   & Ω    & V/A                       \\
    电导          & 西[门子]  & S    & A/V                       \\
    磁通量         & 韦[伯]   & Wb   & V·s                       \\
    磁通量密度;磁感应强度 & 特[斯拉]  & T    & Wb/m\textsuperscript{2}   \\
    电感          & 亨[利]   & H    & Wb/A                      \\
    摄氏温度        & 摄氏度    & ℃    & CDeg                      \\
    光通量         & 流[明]   & lm   & cd·sr                     \\
    光照度         & 勒[克斯]  & lx   & lm/m\textsuperscript{2}   \\
    放射性活度       & 贝可[勒尔] & Bq   & S\textsuperscript{-1}     \\
    吸收剂量        & 戈[瑞]   & Gy   & J/kg                      \\
    剂量当量        & 希[沃特]  & Sv   & J/kg                      \\
    \bottomrule
  \end{tabular}%
  \label{tab:tbl-b3}%
\end{table}%

\hspace*{\fill}\\

\section{非标单位}

\begin{table}[htbp]
  \centering
  \caption{国家选定的非国际单位制单位}
  \begin{longtable}{p{2cm}<{\centering}p{3cm}<{\centering}p{2cm}<{\centering}p{6cm}<{\centering}}
    \toprule
    {量的名称}    & 单位名称    & 单位符号    & 换算关系和说明    \\
    \midrule
    \multirow{3}[2]{*}{时间}   & 分    & min    & 1min=60s    \\
    & [小]时    & h    & 1h=60min3600s    \\
    & 天(日)    & d    & 1d=24h=86400s    \\
    \midrule
    \multirow{3}[2]{*}{平面角} & [角]秒    & ($^{\prime\prime}$) & 1$^{\prime\prime}$=($\pi$/648000)rad    \\
    & [角]分    & ($^{\prime}$)    & 1$^{\prime}$=60$^{\prime\prime}$=($\pi$/1800)rad \\
    & 度    & ($^{\circ}$)    & 1$^{\circ}$=60$^{\prime}$=($\pi$/180)rad    \\
    \bottomrule
  \end{longtable}%
  \label{tab:tbl-b4}%
\end{table}%

\makebox[0.9\textwidth][r]{续表}

\begin{table}[htbp]
  \centering
  \begin{longtable}{p{2cm}<{\centering}p{3cm}<{\centering}p{2cm}<{\centering}p{6cm}<{\centering}}
    \toprule
    {量的名称}    & 单位名称    & 单位符号    & 换算关系和说明    \\
    \midrule
    旋转速度    & 转每分    & r/min    & 1r/min=(1/60)s\textsuperscript{-1}    \\
    \midrule
    长度    & 海里    & n mile    & 1 n mile=1825m(只用于航程)    \\
    \midrule
    速度    & 节    & kn    & 1kn=1 n mile/h=(1852/3600)m/s \\
    \midrule
    \multirow{2}[0]{*}{质量}   & 吨    & t    & 1t=103kg    \\
    & 原子质量单位 & u    & 1u≈1.6605655×10\textsuperscript{-27}kg    \\
    \bottomrule
  \end{longtable}%
\end{table}%

\hspace*{\fill} \\

\section{词头}

\begin{table}[htbp]
  \centering
  \caption{用于构成十进倍数单位的词头}
  \begin{tabular}{ccc}
    \toprule
    所表示的因数     & 词头名称  & 词头符号 \\
    \midrule
    10$^{18}$  & 艾[可萨] & E    \\
    10$^{15}$  & 拍[它]  & P    \\
    10$^{12}$  & 太[拉]  & T    \\
    10$^{9}$   & 吉[咖]  & G    \\
    10$^{6}$   & 兆     & M    \\
    10$^{3}$   & 千     & k    \\
    10$^{2}$   & 百     & h    \\
    10$^{1}$   & 十     & da   \\
    10$^{-1}$  & 分     & d    \\
    10$^{-2}$  & 厘     & c    \\
    10$^{-3}$  & 毫     & m    \\
    10$^{-6}$  & 微     & μ    \\
    10$^{-9}$  & 纳[诺]  & n    \\
    10$^{-12}$ & 皮[可]  & p    \\
    10$^{-15}$ & 飞[母托] & f    \\
    10$^{-18}$ & 阿[托]  & a    \\
    \bottomrule
  \end{tabular}%
  \label{tab:addlabel}%
\end{table}%

\parbox[t]{\textwidth}{
注:1.周、月、年(年的符号为a)为一般常用时间单位;

\hspace{2em}2.{[} {]}内的字,是在不致混淆的情况下,可以省略的字;

\hspace{2em}3.( )内的字为前者的同义语;

\hspace{2em}4.角度单位度分秒的符号不处于数字后时,用括弧;

\hspace{2em}5.升的符号中,小写字母l为备用符号;

\hspace{2em}6.r为``转''的符号;

\hspace{2em}7.人民生活和贸易中,质量习惯称为重量;

\hspace{2em}8.公里为千米的俗称,称号为km;

\hspace{2em}9.10\textsuperscript{4}称为万,10\textsuperscript{8}称为亿,10\textsuperscript{12}称为万亿,这类数词的使用不受词头名称的影响,但不应与词头混淆。
}
