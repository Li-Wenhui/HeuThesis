\chapter{相关标准}[AppendixA]%

A.01 GB 1.1-1993 标准化工作导则。

A.02 GB 7156-1987 文献保密等级代码。

A.03 GB 7713-1987 科学技术报告、学位论文和学术论文的编写格式。

A.04 GB 7714-1987 文后参考文献著录规则。

A.05 GB 15834-1995 标点符号用法。

A.06 GB 3100-1993 国际单位制及其应用。

A.07 GB 3101-1993 有关量、单位和符号的一般原则。

A.08 GB 3102.1-1993 空间和时间的量和单位。

A.09 GB 3102.2-1993 周期及其有关现象的量和单位。

A.10 GB 3102.3-1993 力学的量和单位。

A.11 GB 3102.4-1993 热学的量和单位。

A.12 GB 3102.5-1993 电学和磁学的量和单位。

A.13 GB 3102.6-1993 光及有关电磁辐射的量和单位。

A.14 GB 3102.7-1993 声学的量和单位。

A.15 GB 3102.8-1993 物理化学和分子物理学的量和单位。

A.16 GB 3102.9-1993 原子物理学和核物理学的量和单位。

A.17 GB 3102.10-1993 核反应和电离辐射的量和单位。

A.18 GB 3102.11-1993 物理科学和技术中使用的数学符号。

A.19 GB 3102.12-1993 无量纲参数。

A.20 GB 3102.13-1993 固体物理学的量和单位。

A.21 GB 1434-1978 物理量符号。

A.22 GB4728.1~13-1984.1985 电气图用图形符号。

A.23 GB5465.1-1985 电气设备用图形符号。

A.24 GB5465.2-1985 电气设备用图形符号绘制原则。

A.25 GB7159-1987 电气技术中的文字符号制计通则。

A.26 GB6988-1986 电气制图。

A.27 GB4457-4460-84 机械制图。

A.28 GB131-83 机械制图 表面粗糙度符号、代号及其注法。



\chapter{中华人民共和国法定计量单位}[Units]

我国的法定计量单位(简称法定单位)包括:

(1)国际单位制的基本单位(见表1);

(2)国际单位制的辅助单位(见表2);

(3)国际单位制中具有专门名称的导出单位(见表3);

(4)国家选定的非国际单位制单位(见表4);

(5)由以上单位构成的组合形式的单位;

(6)由词头和以上单位所构成的十进倍数和分数单位(词头见表5)。法定单位的定义、使用方法等,由国家计量局另行规定。 

\hspace*{\fill} \\


\begin{table}[htbp]
  \centering
  \caption{国际单位制的基本单位}
  \begin{tabular}{ccc}
    \toprule
    量的名称  & 单位名称   & 单位称号 \\
    \midrule
    长度    & 米      & m    \\
    质量    & 千克(公斤) & kg   \\
    时间    & 秒      & s    \\
    电流    & 安[培]   & A    \\
    热力学温度 & 开[尔文]  & K    \\
    物质的量  & 摩[尔]   & mol  \\
    发光强度  & 坎[德拉]  & cd   \\
    \bottomrule
  \end{tabular}%
  \label{tab:tbl-b1}%
\end{table}%

\hspace*{\fill} \\


\begin{table}[htbp]
  \centering
  \caption{国际单位制的辅助单位}
  \begin{tabular}{ccc}
    \toprule
    量的名称 & 单位名称 & 单位符号 \\
    \midrule
    平面角  & 弧 度  & rad  \\
    立体角  & 球面度  & sr   \\
    \bottomrule
  \end{tabular}%
  \label{tab:tbl-b2}%
\end{table}%

\hspace*{\fill} \\


\begin{table}[htbp]
  \centering
  \caption{国际单位制中具有专门名称的导出单位}
  \begin{tabular}{cccc}
    \toprule
    量的名称        & 单位名称   & 单位符号 & 其他表示式例                    \\
    \midrule
    频率          & 赫[兹]   & Hz   & S\textsuperscript{-1}     \\
    力;重力        & 牛[顿]   & N    & kg·m/s\textsuperscript{2} \\
    压力;压强;应力    & 帕[斯卡]  & Pa   & N/m\textsuperscript{2}    \\
    能量;功;热      & 焦[耳]   & J    & N·m                       \\
    功率;辐射通量     & 瓦[特]   & W    & J/s                       \\
    电荷量         & 库[仑]   & C    & A·s                       \\
    电位;电压;电动势   & 伏[特]   & V    & W/A                       \\
    电容          & 法[拉]   & F    & C/V                       \\
    电阻          & 欧[姆]   & Ω    & V/A                       \\
    电导          & 西[门子]  & S    & A/V                       \\
    磁通量         & 韦[伯]   & Wb   & V·s                       \\
    磁通量密度;磁感应强度 & 特[斯拉]  & T    & Wb/m\textsuperscript{2}   \\
    电感          & 亨[利]   & H    & Wb/A                      \\
    摄氏温度        & 摄氏度    & ℃    & CDeg                      \\
    光通量         & 流[明]   & lm   & cd·sr                     \\
    光照度         & 勒[克斯]  & lx   & lm/m\textsuperscript{2}   \\
    放射性活度       & 贝可[勒尔] & Bq   & S\textsuperscript{-1}     \\
    吸收剂量        & 戈[瑞]   & Gy   & J/kg                      \\
    剂量当量        & 希[沃特]  & Sv   & J/kg                      \\
    \bottomrule
  \end{tabular}%
  \label{tab:tbl-b3}%
\end{table}%

\hspace*{\fill}\\

\begin{table}[htbp]
  \centering
  \caption{国家选定的非国际单位制单位}
  \begin{longtable}{p{2cm}<{\centering}p{3cm}<{\centering}p{2cm}<{\centering}p{6cm}<{\centering}}
    \toprule
    {量的名称}    & 单位名称    & 单位符号    & 换算关系和说明    \\
    \midrule
    \multirow{3}[2]{*}{时间}   & 分    & min    & 1min=60s    \\
    & [小]时    & h    & 1h=60min3600s    \\
    & 天(日)    & d    & 1d=24h=86400s    \\
    \midrule
    \multirow{3}[2]{*}{平面角} & [角]秒    & ($^{\prime\prime}$) & 1$^{\prime\prime}$=($\pi$/648000)rad    \\
    & [角]分    & ($^{\prime}$)    & 1$^{\prime}$=60$^{\prime\prime}$=($\pi$/1800)rad \\
    & 度    & ($^{\circ}$)    & 1$^{\circ}$=60$^{\prime}$=($\pi$/180)rad    \\
    \bottomrule
  \end{longtable}%
  \label{tab:tbl-b4}%
\end{table}%

\makebox[0.9\textwidth][r]{续表}

\begin{table}[htbp]
  \centering
  \begin{longtable}{p{2cm}<{\centering}p{3cm}<{\centering}p{2cm}<{\centering}p{6cm}<{\centering}}
    \toprule
    {量的名称}    & 单位名称    & 单位符号    & 换算关系和说明    \\
    \midrule
    旋转速度    & 转每分    & r/min    & 1r/min=(1/60)s\textsuperscript{-1}    \\
    \midrule
    长度    & 海里    & n mile    & 1 n mile=1825m(只用于航程)    \\
    \midrule
    速度    & 节    & kn    & 1kn=1 n mile/h=(1852/3600)m/s \\
    \midrule
    \multirow{2}[0]{*}{质量}   & 吨    & t    & 1t=103kg    \\
    & 原子质量单位 & u    & 1u≈1.6605655×10\textsuperscript{-27}kg    \\
    \bottomrule
  \end{longtable}%
\end{table}%

\hspace*{\fill} \\


\begin{table}[htbp]
  \centering
  \caption{用于构成十进倍数单位的词头}
  \begin{tabular}{ccc}
    \toprule
    所表示的因数     & 词头名称  & 词头符号 \\
    \midrule
    10$^{18}$  & 艾[可萨] & E    \\
    10$^{15}$  & 拍[它]  & P    \\
    10$^{12}$  & 太[拉]  & T    \\
    10$^{9}$   & 吉[咖]  & G    \\
    10$^{6}$   & 兆     & M    \\
    10$^{3}$   & 千     & k    \\
    10$^{2}$   & 百     & h    \\
    10$^{1}$   & 十     & da   \\
    10$^{-1}$  & 分     & d    \\
    10$^{-2}$  & 厘     & c    \\
    10$^{-3}$  & 毫     & m    \\
    10$^{-6}$  & 微     & μ    \\
    10$^{-9}$  & 纳[诺]  & n    \\
    10$^{-12}$ & 皮[可]  & p    \\
    10$^{-15}$ & 飞[母托] & f    \\
    10$^{-18}$ & 阿[托]  & a    \\
    \bottomrule
  \end{tabular}%
  \label{tab:addlabel}%
\end{table}%

\parbox[t]{\textwidth}{
注:1.周、月、年(年的符号为a)为一般常用时间单位;

\hspace{2em}2.{[} {]}内的字,是在不致混淆的情况下,可以省略的字;

\hspace{2em}3.( )内的字为前者的同义语;

\hspace{2em}4.角度单位度分秒的符号不处于数字后时,用括弧;

\hspace{2em}5.升的符号中,小写字母l为备用符号;

\hspace{2em}6.r为``转''的符号;

\hspace{2em}7.人民生活和贸易中,质量习惯称为重量;

\hspace{2em}8.公里为千米的俗称,称号为km;

\hspace{2em}9.10\textsuperscript{4}称为万,10\textsuperscript{8}称为亿,10\textsuperscript{12}称为万亿,这类数词的使用不受词头名称的影响,但不应与词头混淆。
}

\chapter{关于出版物上数字用法的规定}[Number Requirement]

国家语言文字工作委员会,国家出版局,国家标准局,国家计量局,国务院办公厅秘书局,中宣部新闻、出版局
(1987年1月1日公布)
为使出版物在涉及数字(如表示时间、长度、重量、面积、容积和其他量值)时使用汉字和阿拉伯数字体例统一,特制定本规定。

1.总的原则

凡是可以使用阿拉伯数字而且又很得体的地方,均应使用阿拉伯数字。遇特殊情形,可以灵活变通,但应力求保持相对统一。重排古籍、出版文学书刊等,仍依照传统体例。

2.应当使用阿拉伯数字的两种主要情况

2.1  公历世纪、年代、年、月、日和时刻

例:公无前8世纪  20世纪80年代  公元前440年  公元7年  1986年10月1日4时20分  4时3刻  下午3点  屈原(约公元前340-前278)扬雄(公元前53-公元18)鲁迅(1881.9.25-1936.10.19)。

注:

\ding{192}~年份不能简写,如1980年不能写作80年,1950-1980年不能写作1950-80年。

\ding{193}~星期几一律用汉字,如星期六。

\ding{194}~夏历和中国清代以前历史纪年用汉字,如正月初五  丙寅年十月十五日  秦文公四十四年(公元前722年)  太平天国庚申十年九月二十四日(清咸丰十年九月二十日,公元1860年11月2日)。

\ding{195}~中国民国纪年和日本年号纪年使用阿拉伯数字,如民国38年(1949年)  昭和16年(1941年)。

2.2  记数与计量(包括正负整数、分数、小数、百分比、约数等)

例:41032 –125.03  1/16  1/1000  4.5倍  34.05\%  4.5\%  3:1  1736.8万公里  4000克  12.5平方米  21.35元  45.6万元  270美元  48岁  10个月  -17℃  0.59安[培]  东经123°50′  维生素12  500多种  60多万公斤  HP-3000型计算机  21/22次特别快车  国家标准GB2312-80  84602部队

注:

\ding{192}~一个数值的书写形式要照顾到上下文。不是出现在一组表示科学计量和具有统计意义数字中的一位数(一、二……、九)可以用汉字,如一个人、三本书、四种产品、六条意见、读了九遍。

\ding{193}~4位和4位以上的数字,采用国际通行的三位分节法。节与节之间空半个阿拉伯数字的位置。非科技专业书刊目前可不分节。但用“,”号分节的办法不符合国际标准和国家标准,应该废止。

\ding{194}~5位以上的数字,尾数零多的,可改写为以万、亿作单位的数。一般情况下,不得以十、百、千、十万、百亿、千亿作单位(千克、千米、千瓦、兆赫等法定计量单位中的词头不在此例)。如:345000000公里可改写为3.45亿公里或34500万公里,不能写作3亿4500万公里或3亿4千5百万公里。

\ding{195}~一个用阿拉伯数字书写的多位数不能移行。

3.应当使用汉字的两种主要情况

3.1  数字作为词素构成定型的词、词组、惯用语、缩略语或具有修辞色彩的语句

例:一律  十滴水  二倍体  三叶虫  八国联军  四氧化三铁  二万五千里长征  第三世界  “一二·九”运动  十月革命  “七五”计划  五省一市  中国工农红军第二方面军  上海二商局 第一书记  路易十六  某部五连二排六班  白发三千丈  相差十万八千里

3.2  邻近的两个数字(一、二……、九)并列连用,表示概数(连用的两个数字之间不应用顿号隔开)

例:二三米  三五天  十三四吨  四十五六岁  七八十种  一千七八百元  五六万套  十之八九。

4.引文标注中版次、卷次、页码,除古籍应与所据版本一致外,一般均使用阿拉伯数字。

例:

\ding{192}~许慎:《说文解字》四部丛刊本卷六上,第九页。

\ding{193}~许慎:《说文解字》中华书局1963年影印陈昌治本,第126页。

\ding{193}~马克思、恩格斯《共产党宣言》,《马克思恩格斯全集》第4卷,人民出版社1958年第1版,第493页。

5.横排标题涉及数字时,可以根据版面实际需要和可能灵活处理。

6.本规定自1987年2月1日起试行。在试用过程中可随时提出意见,以便进一步修订。

\chapter{学位论文结构图}[Paper Struct]

学位论文结构主要包括三部分:前文、正文和后文。其中后文包括附录和结尾等部分可根据实际情况决定是否需要。

\[
  \text{论文结构}
  \begin{cases}
    \text{前文} \\
    \text{正文} \\
    \text{后文}
    \begin{cases}
      \text{附录} \\
      \text{结尾}
    \end{cases}
  \end{cases}
\]

前文部分主要由封面、扉页、摘要、目录等部分组成,本、硕、搏的论文格式要求略有不同,需要根据实际情况进行配置和定义。

\[
  \text{前文}
  \begin{cases}
    \text{封面}           \\
    \text{扉页(中、英文)}     \\
    \text{论文原创性声明}      \\
    \text{摘要及关键词(中、英文)} \\
    \text{目录}           \\
    \text{图表清单(必要时)}
  \end{cases}
\]

正文部分是学位论文的主体,一般第一章是绪论,从第二章开始是论文的具体研究研究内容,可根据情况定义若干章、节和小节等标题。
正文后面部分还包括结论、参考文献、致谢等,根据学位论文类型的不同,所要求的内容也不完全相同,可以根据实际要求进行修改、调整。

\[
  \text{正文}
  \begin{cases}
    \text{绪论(引言)1}               \\
    \text{研究内容\qquad\qquad}
    \begin{cases}
      \text{\quad2}              \\
      \text{\quad3\qquad}
      \begin{cases}
        \text{\quad3.1}           \\
        \text{\quad3.2}           \\
        \text{\quad3.3\quad}
        \begin{cases}
          \text{\quad3.3.1}   \\
          \text{\quad3.3.2\quad}
          \begin{cases}
            \text{\quad3.3.2.1} \\
            \text{\quad3.3.2.2} \\
            \text{\quad3.3.2.3} \\
            \text{\quad.}       \\
            \text{\quad.}       \\
            \text{\quad.}       \\
          \end{cases} \\
          \text{\quad3.3.3}   \\
          \text{\quad.}       \\
          \text{\quad.}       \\
          \text{\quad.}
        \end{cases} \\
        \text{\quad3.4}           \\
        \text{\quad.}             \\
        \text{\quad.}             \\
        \text{\quad.}
      \end{cases} \\
      \text{\quad4}              \\
      \text{\quad.}              \\
      \text{\quad.}              \\
      \text{\quad.}
    \end{cases}   \\
    \text{结论}                    \\
    \text{参考文献}                  \\
    \text{攻读XX学位期间发表的论文和取得的科研成果} \\
    \text{致谢}                    \\
    \text{个人简历(仅对在职人员和在职研究生要求)}  \\
  \end{cases}
\]

后文不是学位论文必须要求的内容,根据实际情况可以将实验数据、程序代码、流程图、公式推导等不宜放入论文正文但又是论文工作不可或缺的内容作为附录放到论文正文后。

所有附录内容可以放在一个~tex~文件中,也可以每个附录定义一个 ~tex~文件。

\[
  \text{附录部分(必要时)}
  \begin{cases}
    \text{附录A}                 \\
    \text{附录B}
    \begin{cases}
      \text{B1}
      \begin{cases}
        \text{B1.1} \\
        \text{b1.2}
      \end{cases} \\
      \text{B2}
    \end{cases} \\
    \text{附录C}                 \\
    \text{......}
  \end{cases}
\]

结尾部分是非必须的,必要时可以加入。

\[
  \text{结尾部分(可以没有)}
  \begin{cases}
    \text{索引} \\
    \text{其他} \\
    \text{......}
  \end{cases}
\]

\chapter{关于论文打印的建议方案}[Print]

论文(含专业学位论文以及涉密学位论文)的版面一律改成A4纸,页边距上、下设置为28mm,左、右设置为25mm,页眉、页脚设置为20mm。装订时上、下、右各切除3mm,学位论文成品版面大小为207mm*291mm。

