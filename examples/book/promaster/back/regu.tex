% !Mode:: "TeX:UTF-8"

\chapter[哈尔滨工程大学研究生学位论文规范]{哈尔滨工程大学研究生学
  位论文\protect\\规范}[Harbin Engineering University Postgraduate Dissertation Writing Specifications]

\section{引言}[Introduction]

学位论文是表明作者从事科学研究取得创造性结果或有了新的见解,并以此为内容撰写而成的学术论文。研究生学位论文展示了研究生在科学研究工作中取得的成果并全面反映了研究生对本学科基础理论和专门知识的掌握程度,是申请和授予相应学位的基本依据。学位论文撰写是研究生培养过程的基本训练之一,必须按照确定的规范认真执行。

本论文规范按照《科学技术报告、学位论文和学术论文的编写格式》(GB 7713-87)、《文后参考文献著录规则》(GB 7714-87)以及《标准化工作导则标准编写的基本规定》(GB/T1.1-1993)制定。

本论文规范适用于我校博士、硕士研究生(工商管理硕士学位论文规范已经规定的内容除外)和以研究生毕业同等学力在我校申请博士、硕士学位的在职人员。博士和硕士学位论文除在字数、理论研究的深度及创造性成果等方面的要求不同以及特殊说明外,对其撰写规范的要求基本一致。


\section{基本要求}[Content specification]

\subsection{撰写依据}[Writing Based]

除论文的语言文字须符合汉语语法规范外,论文撰写应符合国家及各专业部门制定的有关标准。在本规范的“附录A 相关标准”中列出了一些常用标准。

\subsection{论文字数}[Words Required]

博士学位论文,理工类学科:6-8万字,管理及人文学科:8-10万字。

硕士学位论文,理工科:3-4万字,管理及人文学科:4-5万字。

\subsection{论文结构及各部分要求}[Struct Required]

请参见“附录F学位论文结构图”

\subsubsection{前置部分}[Front Parts]

前置部分包括封面、扉页、论文原创性声明、摘要、目录、插图和附表清单。

论文题目要恰当、准确地反映本论文的研究内容。摘要应包括本论文的创造性成果及其理论与实际意义。为了便于国际交流,扉页、摘要、关键词应有中英文两种。插图和附表清单不是必选项,只在图表较多时使用。

\subsubsection{论文主体部分}[Paper Body]

论文主体部分一般包括绪论(引言)、正文、结论、参考文献、攻读学位期间发表的论文和取得的科研成果、致谢、个人简历(个人简历仅对在职人员和在职研究生要求)。

主体部分是学位论文的核心,由于研究工作涉及的学科、选题、研究方法、工作进程、结果表达等有很大差异,故不对主体部分中论文正文内容作统一规定。但要求明确指出本论文的创新点或实际应用之处。文中引用的他人研究成果部分单独书写,并注明出处,不得将其与本人提出的理论分析混淆在一起。论文主体部分要求逻辑清晰,层次分明,实事求是,简练可读。

建议包含以下内容:总体研究方案设计与选择论证、实验和观测方案设计的可行性,有效性和数据处理及分析、理论分析。

\subsubsection{附录部分(必要时)}[Appendix]

对需要收录于学位论文中且又不适合书写于正文中的附加数据、资料、详细公式推导等有助于读者理解学位论文的内容,可作为附录。

\subsubsection{结尾部分(可以没有)}[End Part]

结尾部分可以提供有关输入数据和索引。

\section{具体编写格式}[Format Details]

\subsection{封面}[Cover]

封面是学位论文的外表面,提供应有的信息,并起到保护作用。封面包含以下内容:

a.分类号 \quad 在左上角注明《中国图书资料分类法》的类号和《国际十进分类法UDC》的类号。

b.密级和编号 \quad 若论文内容属保密范围,按国家规定的保密条例,在右上角注明密级,并注明为正本或副本。如系公开发行,不注密级。

c.论文题目 \quad 论文题目名称应恰当、准确地反映本论文的研究内容。学位论文的中文题名不宜超过20字,并尽量不设副标题。题名应标注于封面偏上正中位置,并在其正上方注明“\uline{\qquad}”士学位论文字样(“\uline{\qquad}”指所获硕士或博士学科门类,如管理学硕或管理学博)。若作者为在职人员以同等学力申请学位还应在上述文字和论文题目之间注明“(在职人员)”字样。

d.作者姓名。“

e.导师姓名、专业技术职务 \quad 专业技术职务不可简写。

f.学科、专业名称 \quad 按二级学科填写,若为一级学科博士学位授权专业或该一级学科不设二级学科则按一级学科填写。

g.学位授予单位 \quad 在封面下部居中写明学位授予单位“哈尔滨工程大学”。

\subsection{扉页}[Second Page]

扉页提供整个学位论文有关信息的详细说明,扉页包含封面中的各项内容,并且还包括以下内容:

a.申请学位级别 \quad 应写明学科门类、学位级别

b.论文提交日期

c.论文答辩日期

d.作者所在单位 \quad 本校学生填所在院(系),同等学力申请学位人员
或在职研究生填写本人所在单位。

英文扉页不注密级和编号,其他内容与相应中文内容一致。

\subsection{摘要}[Abstraction]

摘要是学位论文内容的简短陈述,应具有独立性和自含性,即不阅读全文就能获得必要信息。摘要内容要说明研究工作目的、实验方法、结果和最终结论,重点是结果和结论。除实在无变通办法可用以外,摘要中不使用图、表、化学结构式、特殊符号和术语,不标注引用文献号。要求中、英文摘要内容要一致。

\subsection{关键词}[Keywords]

关键词是供检索用的主题词条,应采用能覆盖论文主要内容的通用技术词条(参照相应的技术术语标准)。关键词一般列3-5个,按词条外延层次排列(外延大的排在前面)。英文关键词与中文关键词要求相同。

\subsection{目录}[Content]

目录应包括论文中全部章节的标题及页码,含:
章节题目(要求编到第3级标题,即X.X.X)
参考文献
攻读X士学位期间发表的论文和取得的科研成果
致谢
个人简历(仅对在职人员和在职研究生要求有此要求)
附录(可选项)
索引(可选项)

\subsection{插图和附表清单(可选择)}[Graphics Index]

如果论文中图表较多,可以分别列出清单置于目录之后。图的清单应有序号、图题和页码。表的清单应有序号、表题和页码。

\subsection{章节编排}[Section Arrange]

论文主体部分分章节撰写,每章另起一页。除绪论(引言)外正文每一章后应有一节“本章小结”。绪论(引言)一般作为第一章,结论、参考文献、攻读学位期间发表的论文和取得的科研成果、致谢、个人简历、附录、索引不编排章号。

各章标题要突出重点、简明扼要。字数一般在15字以内,不得使用标点符号。标题中尽量不采用英文缩写词,对必须采用者,应用使用本行业通用缩写词。

章、条的编号参照国家标准GB1.1《标准化工作导则标准编写的基本规定》第8章“标准条文的编排”的有关规定,采用阿拉伯数字分级编号,
即1、1.1、1.1.1……的形式。层次一般不大于四级。

\subsection{绪论(引言)}[Introduce]

绪论(引言)简要说明研究工作的目的、范围、相关领域的前人工作和知识空白、理论基础和分析、研究设想、研究方法和实验设计、预期结果和意义等。

\subsection{正文}[Main Body]


\subsubsection{编码}[Index No]

正文中的图、表、附注、参考文献、公式、算式等一律用阿拉伯数字分章依序连续编排序号或就全文统一依序编排。其标注形式为:图2.1、表3.2;附注1);文献[4];式(3-5)等。

\subsubsection{数字}[Number]

按国家语言学工作委员会等七单位1987年发布的《关于出版物上数字用法的试行规定》,一般采用阿拉伯数字。

\subsubsection{公式}[Fomular]

正文中的公式应居中书写。若公式前有文字,空两格写文字,公式居中写。公式末尾不加标点。

公式序号按章编排,用阿拉伯数字,序号写在右边,并加圆括号,如第一章第一个公式号为“(1-1)”,附录A中的第一个公式为(A1)等。

文中引用公式时,一般用“见式(1-1)”或“由公式(1-1)”。

较长公式必须转行时,只能在等号(=)或加(+)、减(-)、乘()、除(÷)等运算符号后断开转行,上下行尽可能在等号处对齐。

公式中符号的含义和计量单位应注释在公式的下面。每条注释应另行书写,移行时,与其开始写文字的位置对齐。

例:第一章第一个公式:

\begin{equation}\label{form2x1}
  f=\frac{1}{2\pi\sqrt{LC}}
\end{equation}

\begin{tabularx}{\textwidth}{@{}l@{\quad}r@{——}X@{}}
式中 & $\boldsymbol{f}$    & 频率,MHz; \\
	 & $\boldsymbol{L}$    & 电感量,H;  \\
 	 & $\mathbf{C}$        & 电容量,pF。                       
\end{tabularx}\vspace{3.15bp}

\subsubsection{插图}[Pictures]

插图包括曲线图、构造图、示意图、图解、框图、流程图、记录图、布置图、地图、照片等。

第一图应有简短确切的题名,连同图序置于图下,图名中不允许使用标点符号,图名后不加标点符号。图序与图名之间空一格。博士学位论文还应在中文图名下注明相应的英文图名。

插图应与正文的内容紧密配合,插图和有关图形符号符合制图、图形符等有关标准规定。插图应具有“自明性”,即只看图、图题和图例,不阅读正文就可理解图意。必要时,应将图上的符号、标记、代码以及实验条件等,用最简练的文字,横排于图题下方,作为图例说明。

插图的纵横坐标必须标注“量、标准规定符号、单位”。此三者只有在不必要标明(如无量纲等)的情况下方可省略。坐标上标注的量的符号和缩略词必须与正文中的一致。

表示函数关系的曲线图,如有确定曲线的函数式,则应在有关条文中,或在图的下方,或在图中适当位置写出。曲线图内,不应有过多的空白,如果曲线不占其整个面积,应当将图截短,只保留有曲线的坐标部分。

若条件允许,插图要用相应的计算机绘图软件绘制。

每一图应有简短确切的题名,连同图号置于图下。

\subsubsection{插表}[Table]

表序与表名书写于表的正上方,表序与表名之间空一格,表名不允许使用标点符号,表名后不加标点。博士学位论文还应在表名下注明相应的英文表名。

表格的上部和下部用粗实线闭合,左、右两侧不加竖线闭合。在表格横向狭而长,排版时幅面宽度不够时,可将表格分为两段,用细双线接排在一页内。(见\ref{tab:tbl-1-1} - \ref{tab:tbl-1-3})。

表格中各栏参数的计量单位相同时,应将单位写在表的右上角(见表3.1);如计量单位不同时,应将单位分别写在各栏参数名称的下方。若相邻参数采用相同的单位时,可合并写在它们共同的单位栏内(见表3.2);如表格中大多数的计量单位相同,可将该单位写在右上角,将其余的少数单位写在有关栏内(见表3.3)。

当插表太宽,无法在该页横排时,可以逆时针方向旋转90度放置。

% Table generated by Excel2LaTeX from sheet 'Sheet1'
\begin{table}[htbp]
  \centering\wuhao
  \caption{横向长标示例}\hspace{100mm}mm\\
  \begin{tabular}{p{4.0em}cccccc}
    \toprule
    \multicolumn{1}{c}{} & \multicolumn{1}{p{4.0em}}{a}  & \multicolumn{1}{p{4.0em}}{b} & \multicolumn{1}{p{4.0em}}{c} & \multicolumn{1}{p{4.0em}}{d} & \multicolumn{1}{p{4.0em}}{e} & \multicolumn{1}{p{4.0em}}{f} \\
    \midrule
    A                    &                               &                              &                              &                              &                              &                              \\
    B                    &                               &                              &                              &                              &                              &                              \\
    \midrule
    \midrule
    \multicolumn{1}{c}{} & \multicolumn{1}{p{4.0em}}{g } & \multicolumn{1}{p{4.0em}}{h} & \multicolumn{1}{p{4.0em}}{i} & \multicolumn{1}{p{4.0em}}{j} & \multicolumn{1}{p{4.0em}}{k} & \multicolumn{1}{p{4.0em}}{l} \\
    \midrule
    A                    &                               &                              &                              &                              &                              &                              \\
    B                    &                               &                              &                              &                              &                              &                              \\
    \bottomrule
  \end{tabular}%
  \label{tab:tbl-1-1}%
\end{table}%

% Table generated by Excel2LaTeX from sheet 'Sheet2'
\begin{table}[htbp]
  \centering\wuhao
  \caption{表格示例1}
  \vspace{0.5em}
  \begin{tabular}{cccccccc}
    \toprule
    \multicolumn{1}{c}{\multirow{2}[4]{*}{m/t}} & \multicolumn{1}{c}{\multirow{2}[4]{*}{H/m}} & \multicolumn{1}{p{4.0em}}{B}    & \multicolumn{1}{p{4.0em}}{K}  & \multicolumn{1}{p{4.0em}}{L} & \multicolumn{1}{p{4.0em}}{H1} & \multicolumn{1}{p{4.0em}}{H} & \multicolumn{1}{p{4.0em}}{轮压} \\
    \cmidrule{3-8}                              &                                             & \multicolumn{5}{p{20.95em}}{Mm} & \multicolumn{1}{p{4.0em}}{Pa}                                                                                                                                 \\
    \midrule
                                                &                                             & \multicolumn{5}{c}{}            &                                                                                                                                                               \\
    \bottomrule
  \end{tabular}%
  \label{tab:tbl-1-2}%
\end{table}%

% Table generated by Excel2LaTeX from sheet 'Sheet3'
\begin{table}[htbp]
  \centering\wuhao
  \caption{表格示例2}
  \vspace{0.5em}
  \begin{tabular}{p{2.6em}p{2.6em}p{2.6em}ccccp{3.2em}}
    \toprule
    \multirow{2}[4]{*}{规格} & \multirow{2}[4]{*}{A} & \multirow{2}[4]{*}{B} & \multicolumn{2}{c}{H} & \multicolumn{2}{c}{L} & \multirow{2}[4]{*}{m/Kg}                    \\
    \cmidrule{4-7}           &                       &                       & 基本尺寸              & 极限偏差              & 基本尺寸                 & 极限偏差 &       \\
    \midrule
    100                      & 100                   & -                     & 4                     &                       & 10                       & ±0.3     & 8.5   \\
    200                      & 150                   & 20                    & 6                     & ±0.1                  & 50                       & ±0.4     & 23.5  \\
    300                      & 200                   & 30                    & 10                    &                       & 70                       & ±0.5     & 75.6  \\
    400                      & -                     & 40                    & 15                    &                       & 100                      & ±.06     & 120.7 \\
    \bottomrule
  \end{tabular}%
  \label{tab:tbl-1-3}%
\end{table}%

表格中各栏参数的计量单位相同时,应将单位写在表的右上角(见表\ref{tab:tbl-1-1});如计量单位不同时,应将单位分别写在各栏参数名称的下方。若相邻参数采用相同的单位时,可合并写在它们共同的单位栏内(见表3.2);如表格中大多数的计量单位相同,可将该单位写在右上角,将其余的少数单位写在有关栏内(见表3.3)。

当插表太宽,无法在该页横排时,可以逆时针方向旋转90度放置。

\subsubsection{文献引用形式}[Reference Style]

引用文献标示应置于所引内容最末句的右上角,用小五号字体。所引文献编号用阿拉伯数字置于方括号“[]”中。当提及的参考文献为文中直接说明时,文献编号置于方括号中与正文排齐,字号大小与正文相同。

\subsubsection{科技术语和缩略词}[Terms and abbreviations]

科技术语和缩略词应采用国家标准。标准中未规定的要按本学科或本专业的权威性机构或学术团体公布的规定执行。全文名词术语必须统一。首次出现的特殊科技术语和缩略词应在适当位置加以说明或注解。

\subsubsection{物理量名称和符号}[Physical quantity name and symbol]

物理量名称和符号应符合国标GB1434-78《物理量符号》、GB3100-82《国际单位制及其应用》、GB3101-82《有关量、单位和符号的一般原则》、GB3102.1-1993到GB3102.13-1993(名称请见附录A)的规定,论文中某一物理量的名称符号应统一。

\subsubsection{物理量计量单位(打印)}[Physical quantity UOM]

计量单位及符号必须采用1984年2月27日国务院发布的《中华人民共和国法定计量单位》并遵照《中华人民共和国法定计量单位使用方法》执行。

\subsubsection{外文字母的正、斜体用法}[外文字母的正、斜体用法]

按照GB3100~3102-1993(名称请见附录A)及GB7159-87 电气技术中的文字符号设计通则的规定使用,即物理量符号、物理常量、变量符号用斜体,计量单位等符号均用正体。$sin{x}$、$cos{x}$等三角函数应用正体。

\subsection{结论}[conclusion]

结论是论文最终的、总体的结论,不是正文中各段的小结的简单重复。
在结论中应明确指出本研究内容的创造性成果或创新点理论(含新见解、新观点),对其应用前景和社会、经济价值等加以预测和评价,及其今后进一步进行研究工作的展望与设想。结论应该准确、完整、明确、精练。

\subsection{致谢}[thanks]

致谢内容应该简洁明了、实事求是。

\subsection{参考文献}[reference]

参考文献书写格式应符合GB7714-87《文后参考文献著录规则》。若同一文献中有多处被引用,则要写出相应引用页码,各起止页码间空一格,排列按引用顺序,不按页码顺序。常用参考文献编写项目和顺序规定如下,(其中版次为第一版的要省略版次):

著作图书文献

序号 \quad 作者.书名.版次.出版者,出版年:引用部分起止页

翻译图书文献

序号 \quad 作者.书名.译者.版次.出版者,出版年:引用部分起止页

学术刊物文献

序号 \quad 作者.文章名.学术刊物名.年,卷(期):引用部分起止页

学术会议文献

序号 \quad 作者.文章名.编者名.会议名称,会议地址,年份.出版地:出版者,出版年:引用部分起止页

学位论文类参考文献

序号 \quad 研究生名.学位论文题目.学校及学位论文级别.答辩年份:引用部分起止页

学术会议若出版论文集者,可在会议名称后加上“论文集”字样。未出版论文集者省去“出版者”、“出版地”、“出版年”三项。会议地址与出版地相同者省略“出版地”。会议年份与出版年相同者省略“出版年”。

产品说明书、各类标准、各种报纸上刊登的文章及未公开发表的研究报告不宜作为参考文献。

\subsection{攻读XX学位期间发表的论文和取得的科研成果}[Publications]

攻读学位期间发表的论文格式与参考文献相同。取得的科研成果格式为:

a.获奖项目名称 \quad 奖励名称 \quad 级别 \quad 日期 \quad 排名

b.获专利名称 \quad 专利号 \quad 专利所在国家和专利类别 \quad 日期 \quad 排名

\textbf{注意:不论有无发表的论文和取得的科研成果,本章不能省略。}

\subsection{个人简历}[Resume]

仅对同等学力申请学位人员和在职研究生要求。

个人简历应包含取得各级学位的时间和地点、担任各种职务的时间和任职单位、参加过的科研工作和取得的成果。

\subsection{附录}[Appendix]

附录是论文主体部分的补充项目,视论文需要决定是否使用。

对需要收录于学位论文中,但又不便书写于正文中的附加数据、资料、详细公式推导等有特色的内容,可做为附录。每一附录均另页起,论文的附录依序用大写正体A,B,C,……编序号,如:附录A。附录中的图、表、式、参考文献等另行编序号,与正文分开,也一律用阿拉伯数字编码,但在数码前冠以附录序码,如:图A1;表B2;式(B3);文献[A5]等。

\subsection{索引}[Index]

为便于检索文中内容,可编制索引置于论文之后,索引不是必需的项目。索引以论文中的专业词语为检索线索,指出其相关内容的所在页码。索引用中、英两种文字分别书写,中文在前。中文按各词汉语拼音第一个字母排序,英文按该词第一个英文字母排序。

\section{打印要求}[Print Requirement]

研究生学位论文一律要求用计算机打印,并尽量用流行的排版软件。

我校各级研究生学位论文(含专业学位论文以及涉密学位论文)的版面一律改成A4纸,页边距上、下设置为28mm,左、右设置为25mm,页眉、页脚设置为20mm。装订时上、下、右各切除3mm,学位论文成品版面大小为207mm*291mm。

\subsection{字体与字号}[Font and Size]

各章题序及标题:小2号黑体;

各节的一级题序及标题:小3号黑体;

各节的二级题序及标题:4号黑体;

各节的三级题序及标题:小4号黑体;

款、项:均采用小4号黑体;

正文用小4号宋体。

摘要、结论、参考文献、致谢、攻读学位期间发表的论文和取得的科研成果、个人简历等部分按章处理,即标题小2号黑体,内容小4号宋体。目录的标题采用小2号黑体,内容中章的标题用小4号黑体,其它为小4号宋体。

\subsection{页码}[Page Number]

论文页码一律用阿拉伯数字连续编码。页码由第1章的首页开始,作为第1页。论文前置部分(封面、扉页、摘要、目录、插图和附表清单)不编排页码。页码位于整页下部居中。封面、扉页为右页且为单面页。摘要、目录、插图和附表表单、各章、结论、参考文献、攻读xx学位期间发表的论文和取得的科研成果、致谢、个人简历另起一页。

\subsection{页眉与页脚}[Page Head and Foot]

论文不加页脚。论文的封面、扉页、不加页眉,其它部分均加页眉。页眉采用宋体5号字居中放置,若论文为双面印刷,则奇、偶页页眉内容不同。奇数页为本页内容所属的章的题目,即“第n章 ”的形式,若本页含的内容不编章号,则页眉为本学位论文的题目;偶数页为“哈尔滨工程大学 \underline{\hspace{1em}} 士学位论文”( \underline{\hspace{1em}} 为“硕”或“博”)。若论文为单面印刷,则奇、偶页内容相同一律为“哈尔滨工程大学 \underline{\hspace{1em}} 士学位论文”。页眉下划线要求为双线,上细下粗。细线粗约0.5mm,粗线约0.8mm,粗线与细线间距约为0.3mm。

\subsection{封面}[Cover]


\subsection{扉页(内封)}[Second Page]


\subsection{摘要及关键词}[Abstract and Key Words]

摘要题头应居中,然后隔行书写摘要的文字部分。摘要文字之后隔一行写关键词,格式见样例。

\subsection{印刷和装订}[Print and Binging]

硕士、博士学位论文原件要求与计算机程序清单、实验原始记录等与论文有关的材料装订在一起。

硕士学位论文的封面采用“绿丝棉”纸,博士学位论文封面均采用“莱妮纹”纸(克重:200gsm,颜色:银灰色)。

书脊处应印刷论文题目及“哈尔滨工程大学 \underline{\hspace{1em}} 士学位论文”(\underline{\hspace{1em}} 为“博”或“硕”)字样,字体用适当字号的宋体字。



% Local Variables:
% TeX-master: "../thesis"
% TeX-engine: xetex
% End: