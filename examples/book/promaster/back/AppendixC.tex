
\appendix{C}{关于出版物上数字用法的规定}%[Number Requirement]

国家语言文字工作委员会,国家出版局,国家标准局,国家计量局,国务院办公厅秘书局,中宣部新闻、出版局
(1987年1月1日公布)
为使出版物在涉及数字(如表示时间、长度、重量、面积、容积和其他量值)时使用汉字和阿拉伯数字体例统一,特制定本规定。

1.总的原则

凡是可以使用阿拉伯数字而且又很得体的地方,均应使用阿拉伯数字。遇特殊情形,可以灵活变通,但应力求保持相对统一。重排古籍、出版文学书刊等,仍依照传统体例。

2.应当使用阿拉伯数字的两种主要情况

2.1  公历世纪、年代、年、月、日和时刻

例:公无前8世纪  20世纪80年代  公元前440年  公元7年  1986年10月1日4时20分  4时3刻  下午3点  屈原(约公元前340-前278)扬雄(公元前53-公元18)鲁迅(1881.9.25-1936.10.19)。

注:

\ding{192}~年份不能简写,如1980年不能写作80年,1950-1980年不能写作1950-80年。

\ding{193}~星期几一律用汉字,如星期六。

\ding{194}~夏历和中国清代以前历史纪年用汉字,如正月初五  丙寅年十月十五日  秦文公四十四年(公元前722年)  太平天国庚申十年九月二十四日(清咸丰十年九月二十日,公元1860年11月2日)。

\ding{195}~中国民国纪年和日本年号纪年使用阿拉伯数字,如民国38年(1949年)  昭和16年(1941年)。

2.2  记数与计量(包括正负整数、分数、小数、百分比、约数等)

例:41032 –125.03  1/16  1/1000  4.5倍  34.05\%  4.5\%  3:1  1736.8万公里  4000克  12.5平方米  21.35元  45.6万元  270美元  48岁  10个月  -17℃  0.59安[培]  东经123°50′  维生素12  500多种  60多万公斤  HP-3000型计算机  21/22次特别快车  国家标准GB2312-80  84602部队

注:

\ding{192}~一个数值的书写形式要照顾到上下文。不是出现在一组表示科学计量和具有统计意义数字中的一位数(一、二……、九)可以用汉字,如一个人、三本书、四种产品、六条意见、读了九遍。

\ding{193}~4位和4位以上的数字,采用国际通行的三位分节法。节与节之间空半个阿拉伯数字的位置。非科技专业书刊目前可不分节。但用“,”号分节的办法不符合国际标准和国家标准,应该废止。

\ding{194}~5位以上的数字,尾数零多的,可改写为以万、亿作单位的数。一般情况下,不得以十、百、千、十万、百亿、千亿作单位(千克、千米、千瓦、兆赫等法定计量单位中的词头不在此例)。如:345000000公里可改写为3.45亿公里或34500万公里,不能写作3亿4500万公里或3亿4千5百万公里。

\ding{195}~一个用阿拉伯数字书写的多位数不能移行。

3.应当使用汉字的两种主要情况

3.1  数字作为词素构成定型的词、词组、惯用语、缩略语或具有修辞色彩的语句

例:一律  十滴水  二倍体  三叶虫  八国联军  四氧化三铁  二万五千里长征  第三世界  “一二·九”运动  十月革命  “七五”计划  五省一市  中国工农红军第二方面军  上海二商局 第一书记  路易十六  某部五连二排六班  白发三千丈  相差十万八千里

3.2  邻近的两个数字(一、二……、九)并列连用,表示概数(连用的两个数字之间不应用顿号隔开)

例:二三米  三五天  十三四吨  四十五六岁  七八十种  一千七八百元  五六万套  十之八九。

4.引文标注中版次、卷次、页码,除古籍应与所据版本一致外,一般均使用阿拉伯数字。

例:

\ding{192}~许慎:《说文解字》四部丛刊本卷六上,第九页。

\ding{193}~许慎:《说文解字》中华书局1963年影印陈昌治本,第126页。

\ding{193}~马克思、恩格斯《共产党宣言》,《马克思恩格斯全集》第4卷,人民出版社1958年第1版,第493页。

5.横排标题涉及数字时,可以根据版面实际需要和可能灵活处理。

6.本规定自1987年2月1日起试行。在试用过程中可随时提出意见,以便进一步修订。
